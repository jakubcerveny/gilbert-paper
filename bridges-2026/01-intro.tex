\section*{Generalized Hilbert Curves}

In a website application and scanned note \cite{lutanho2003}, Tautenhahn provided the basis for a generalized 2D
space filling curve.
Tautenhahn also included policies for when to subdivide regions preferentially in only one dimension
that help to create more \textit{harmonious} curve realizations.
Tautenhahn's exploration details the parity arguments necessary for when a subdivision scheme can be employed
without creating diagonal moves (\textit{notches}).

In this paper, we extend Tautenhahn's ideas to create a 2D and 3D generalized Hilbert curve.
We further extend Tautenhahn's core ideas on when to use alternate subdivision schemes when one length is much
larger than the rest, what we call \textit{eccentric cases}, and extend them to 3D.
Tautenhahn's unpublished reasoning behind the constants used in the 2d eccentric split case are briefly discussed
later in this paper \footnote{ Through personal communication with the authors, Tautenhahn
kindly provided the reasoning for the constants used in the eccentric split }.

\begin{figure}[!htb]

  \begin{minipage}{0.48\textwidth}

    \centering
    \includegraphics[width=\linewidth]{gilbert2d_examples.pdf}
    \caption{ 2D Gilbert curves for i) $8 \times 8$, ii) $18 \times 6$, iv) $13 \times 8$ (with notch), iv) $14 \times 14$ }
    \label{fig:gilbert2d_examples}
  \end{minipage}\hfill
  \begin{minipage}{0.48\textwidth}
    \centering
    \includegraphics[width=\linewidth]{gilbert3d_orig_examples.pdf}
    \caption{ 3D Gilbert curves for i) $4 \times 4 \times 4$, ii) $6 \times 6 \times 6$, iii) $8 \times 4 \times 4$, iv) $5 \times 4 \times 4$ (with notch) }
    \label{fig:examples3d}

  \end{minipage}

\end{figure}


\section*{Valid Paths from Grid Parity}

\begin{figure}[!htb]

  \begin{minipage}{0.48\textwidth}

    \centering
    %\begin{table}[h]
    %  \centering
      \begin{tabular}[t]{cr|cc}
        \multicolumn{2}{c}{ \multirow{2}{*}{Path Possible} } & \multicolumn{2}{c}{Volume} \\
        & & \textit{even} & \textit{odd} \\
        \hline
          %\multirow{2}{*}{$|\alpha| \bmod 2$} & \textit{even} & Yes & Yes \\
          \multirow{2}{*}{ $|\alpha|$ } & \textit{even} & Yes & Yes \\
           & \textit{odd} & \textbf{No} & Yes \\
         \hline
      \end{tabular}
      \caption{ $|\alpha|$ is the distance of endpoints.
                A Hamiltonian path is possible only when $|\alpha|$ is even or both
                $|\alpha|$ and the volume are odd. }
      \label{table:pathTable}
    %\end{table}

  \end{minipage}\hfill
  \begin{minipage}{0.48\textwidth}

    \centering
    \includegraphics[width=\linewidth]{simple_hampath.pdf}
    \caption{ Examples of Hamiltonian paths for small grid sizes. A red `x' corresponds to no possible path for the chosen endpoints. }
    \label{fig:exampleHampath}

  \end{minipage}


\end{figure}

The feasibility of determining whether there exists a Hamiltonian path connecting endpoints on the
corners in a rectangular cuboid
grid region can be accomplished through parity arguments.
Label grid cell points in a volume as 0 or 1,
alternating between labels with every axis-aligned single step move.
Any Hamiltonian path that ends at one of the three remaining corners has to have the same parity as the starting point if the
volume is odd, or different parity if the volume is even.


For a path starting at $(0,0,0)$ and ending $|\alpha|$ steps in one of the axis-aligned dimensions,
then Table \ref{table:pathTable} enumerates this condition under which a valid path is possible.
Figure \ref{fig:exampleHampath} illustrates this for starting position $(0,0)$ with areas $(2 \times 2)$, $(3 \times 2)$ and $(3 \times 3)$,
where a red cross indicating a precluded endpoint.

Without loss of generality, we will assume a curve starts from position $p_s=(0,0,0)$ and has proposed
endpoint at $p_e=((w-1),0,0)$, with a cuboid region as $\alpha = (w,0,0), \beta = (0,h,0), \gamma = (0,0,d)$.
We state, without proof, that
a Hamiltonian path is always possible from $p_s$ to $p_e$ when $|\alpha|$ is even or when all of $|\alpha|$, $|\beta|$ and $|\gamma|$
are odd  $(|\alpha| \cdot (1 - |\beta| \cdot |\gamma|) \equiv 0 \bmod 2)$.

When the condition $(|\alpha| \cdot (1 - |\beta| \cdot |\gamma|) \equiv 0 \bmod 2)$ is met ($|\alpha|$ even or all of $|\alpha|, |\beta|, |\gamma|$ odd)
any cuboid subdivision will always have a Hamiltonian path feasible within it.
We can recreate a Hamiltonian path in the larger cuboid region by connecting neighboring endpoints in each of the adjacent subdivided regions.

For cuboids that violate this condition, there will be at least one notch.
The 2D Gilbert curve limits the number of notches to one.
The 3D Gilbert curve's subdivision strategy creates a notch when the distance between endpoints is odd,
potentially creating more than one notch.

In both the 2D and 3D case, when the original side lengths are all even, no notches will be present.

\section*{2D Gilbert Curve Algorithm}

\begin{figure}[!htb]

  \begin{minipage}{0.55\textwidth}
    \centering
    \includegraphics[width=\linewidth]{gilbert2d_mainsubdiv.pdf}
    \caption{ Subdivision strategy for the 2D Gilbert curve. }
    \label{fig:main2dsubdiv}
  \end{minipage}\hfill
  \begin{minipage}{0.45\textwidth}
    \centering
    \includegraphics[width=\linewidth]{gilbert3d_explode.pdf}
    \caption{ The main subdivision strategy for the 3D Gilbert curve. }
    \label{fig:gilbert3DJSplit}
  \end{minipage}

\end{figure}

\begin{figure*}[ht]
  \centering
  \includegraphics[width=\linewidth]{config_production.pdf}
  \caption{ Enumeration of the subdivision template depending on different parities of $\alpha$ and $\beta$ dimensions. }
  \label{fig:production2d}
\end{figure*}


Algorithm \ref{alg:gilbert2d} shows the pseudo-code for computing the 2D Gilbert curve.
Note that $\alpha$ and $\beta$ are taken to be vectors in 3D, where the third dimension
can be ignored if a purely 2D curve is desired.
The generalization to 3D allows the Gilbert2D function to be used unaltered when the
3D Gilbert curve needs to trace out in-plane sub-curves.

The $\delta(\cdot)$ function returns one of the six directional vectors indicating which of
the major signed axis aligned directions the input vector points to ($(\pm1,0,0), (0,\pm1,0),(0,0,\pm1)$).

Algorithm \ref{alg:gilbert2d} assumes standard Euclidean two norm ($|v| = \sqrt{v_0^2 + v_1^2 + v_2^2}$)
and abuses notation by allowing scalar to vector multiplication ($i \in \mathbb{Z}, v \in \mathbb{Z}^3, i \cdot v \to ( i \cdot v_0, i \cdot v_1, i \cdot v_2 )$).
where the context is clear.


\begin{minipage}[ht]{0.48\linewidth}
\begin{algorithm}[H]
  \begin{algorithmic}

    \State \textit{\# $p, \alpha, \beta \in \mathbb{Z}^3$}
    \Function{Gilbert2D}{$p$, $\alpha$, $\beta$}
      \State $\alpha_2, \beta_2  = \text{div}(\alpha, 2), \text{div}(\beta, 2)$
      \If{ $(|\beta| \equiv 1)$ }
        \State \textbf{yield} $p + i \cdot \delta(\alpha)$ \textbf{forall} $i \in |\alpha|$
      \ElsIf{ $(|\alpha| \equiv 1)$ }
        \State \textbf{yield} $p + i \cdot \delta(\alpha)$ \textbf{forall} $i \in |\beta|$
      \ElsIf{ $(2 |\alpha| > 3 |\beta|)$ }
        \If{ $(|\alpha_2| > 2)$ and $(|\alpha_2| \bmod{2} \equiv 1)$ }
          \State $\alpha_2 \leftarrow \alpha_2 + \delta(\alpha)$
        \EndIf
        \State \textbf{yield} Gilbert2D($p$, $\alpha_2$, $\beta$)
        \State \textbf{yield} Gilbert2D($p + \alpha_2$, $\alpha - \alpha_2$, $\beta$)
      \Else
        \If{ $(|\beta_2| > 2)$ and $(|\beta_2| \bmod{2} \equiv 1)$ }
          \State $\beta_2 \leftarrow \beta_2 + \delta(\beta)$
        \EndIf
        \State \textbf{yield} Gilbert2D($p$, \\ \hskip9.75em $\beta_2$, $\alpha_2$)
        \State \textbf{yield} Gilbert2D($p + \beta_2$, \\ \hskip9.75em $\alpha$, $(\beta - \beta_2)$)
        \State \textbf{yield} Gilbert2D($p + \alpha - \delta(\alpha) + \beta_2 - \delta(\beta)$, \\ \hskip9.75em $\beta_2$, $-(\alpha - \alpha_2)$)
      \EndIf
    \EndFunction

  \end{algorithmic}
\end{algorithm}
\end{minipage}\hfill
\begin{minipage}[ht]{0.48\linewidth}
\begin{algorithm}[H]
  \begin{algorithmic}

    \State \textit{\# $p, \alpha, \beta, \gamma \in \mathbb{Z}^3$}
    \Function{Gilbert3D}{$p$, $\alpha$, $\beta$, $\gamma$}

    \State
    \State\textit{\# Parity of dimensions}

    \State $\alpha_0 \leftarrow (|\alpha|\bmod{2})$
    \State $\beta_0 \leftarrow (|\beta|\bmod{2})$
    \State $\gamma_0 \leftarrow (|\gamma|\bmod{2})$

    \State
    \State\textit{\# Base cases }
    \State \textbf{if} ($(|\alpha|\equiv 2)$ and $(|\beta|\equiv 2)$ and $(|\gamma| \equiv 2)$)
    \State \hskip1.5em \Return Hilbert3D($p$,$\alpha$,$\beta$,$\gamma$)
    \State \Return Gilbert2D($p$,$\beta$,$\gamma$) \textbf{if} $(|\alpha| \equiv 1)$
    \State \Return Gilbert2D($p$,$\alpha$,$\gamma$) \textbf{if} $(|\beta| \equiv 1)$
    \State \Return Gilbert2D($p$,$\alpha$,$\beta$) \textbf{if} $(|\gamma| \equiv 1)$

    \State
    \State\textit{\# Eccentric cases }

    \State \textbf{if }$(3 |\alpha|>5|\beta|) \text{ and } (3|\alpha|>5|\gamma|))$
    \State \hskip1.5em \Return $S _ 0$($p$,$\alpha$,$\beta$,$\gamma$)
    \State \textbf{if }$(2 |\beta| > 3 |\gamma|) \text{ or }(2 |\beta| > 3 |\alpha|))$
    \State \hskip1.5em \Return $S _ 2$($p$,$\alpha$,$\beta$,$\gamma$)
    \State \textbf{if }$(2 |\gamma| > 3 |\beta|)$
    \State \hskip1.5em \Return $S _ 1$($p$,$\alpha$,$\beta$,$\gamma$)

    \State
    \State \textit{\# Bulk recursion }
    \State \Return $J _ 0$($p$,$\alpha$,$\beta$,$\gamma$) \textbf{if} $(\gamma_0 \equiv 0)$
    \State \Return $J _ 1$($p$,$\alpha$,$\beta$,$\gamma$) \textbf{if} $(\alpha_0 \equiv 0)$or$(\beta_0\equiv 0)$
    \State \Return $J _ 2$($p$,$\alpha$,$\beta$,$\gamma$)

    \EndFunction

  \end{algorithmic}
\end{algorithm}
\end{minipage}




