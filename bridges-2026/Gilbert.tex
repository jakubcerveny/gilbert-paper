\documentclass[letterpaper,11pt]{article}
\usepackage{amsmath, amsthm, amssymb}
\usepackage{bridges}
\usepackage{graphicx}

\usepackage[colorlinks=true, urlcolor=blue, citecolor=black, linkcolor=black]{hyperref}
%\usepackage{subcaption}

\usepackage{algorithm}
\usepackage{algpseudocode}
%\usepackage{hyperref} % already loaded above

%\usepackage{svg}
%\usepackage{multirow}
%\usepackage{caption}
%\usepackage{tikz}
%\usepackage{transparent}

%\usepackage[bottom]{footmisc}

\urlstyle{rm}

\newtheorem{lemma}{Lemma}

%\sloppy
\usepackage{listings}
%\lstset{breaklines=true}

\title{ A Generalized 2D and 3D Hilbert Curve }
\author{ Jakub \v{C}erven\'{y} and Zzyv Zzyzek }

\date{}

\begin{document}

\maketitle

\thispagestyle{empty}

\begin{abstract}
  Hilbert curves are classic space-filling curves with strong locality properties, widely
  used for linearizing 2D and 3D data. In their discrete form, however, the standard
  Hilbert construction applies most naturally to square or cubic grids with side lengths
  that are exact powers of two.
  We present the \emph{Gilbert curve}, a simple recursive, Hilbert-like discrete space-filling
  curve that produces Hamiltonian paths on all even-sized 2D rectangles,  all even-sized 3D rectangular cuboids,
  and relaxed traversals on odd-sized domains.
%  curve that produces Hamiltonian paths on all even-sized 2D rectangles and 3D rectangular cuboids,
%  and relaxed traversals on odd-sized domains.
\end{abstract}


\section*{Introduction}

Space-filling curves map a one-dimensional ordering to a multi-dimensional grid while
preserving locality.
Discrete Hilbert curves, in particular, can be
used to linearize images or volumes for cache-coherent traversal,
construct spatial indexes, visualize data layout, and more \cite{anders2009, bohm2021, cortesi2011, moon2001, munroe2006}.
In practical settings, however, underlying grids might not be square or cubic with
power-of-two side lengths. Images and volumes can come in arbitrary dimensions.
Tiling, padding, or resampling to fit a power-of-two Hilbert curve introduces overhead
and can harm locality near boundaries.

\begin{figure}[!htb]
  \begin{minipage}{0.48\textwidth}
    \centering
    \includegraphics[width=\linewidth]{gilbert2d_examples.pdf}
    \caption{2D Gilbert curves for i) $8\times 8$, ii) $18\times 6$, iii) $13\times 8$ (with diag. move), iv) $14\times 14$.}
    \label{fig:gilbert2d_examples}
  \end{minipage}\hfill
  \begin{minipage}{0.46\textwidth}
    \centering
    \includegraphics[width=\linewidth]{gilbert3d_orig_examples.pdf}
    \caption{3D Gilbert curves for i) $4 \times 4 \times 4$, \\ ii) $6 \times 6 \times 6$, iii) $8 \times 4 \times 4$, \\ iv) $5  \times 4 \times 4$ (with diag. move).}
    \label{fig:examples3d}
  \end{minipage}
\end{figure}


This paper presents the \emph{Gilbert curve}, a recursive construction that produces
paths for 2D rectangles and 3D cuboids.
Our goal is a path construction that is easy to implement, has a uniform recursive structure,
and reduces to a Hilbert curve in the power-of-two square/cube cases.
In this paper, we:

\begin{itemize}
  \item Describe a simple recursive construction of Hilbert-like paths on arbitrary 2D rectangular grids
  \item Formalize a parity condition on corner endpoints and use it to guide valid recursive subdivisions
  \item Extend the same construction principles naturally to arbitrary 3D cuboid grids
\end{itemize}

In certain conditions, combinations of side length sizes and the parity of path endpoints
preclude a strict non self-intersecting path that visits every cell exactly once (a \textit{Hamiltonian path})
without any diagonal steps.
We will briefly address the conditions under which the Gilbert curve algorithm introduces diagonal steps throughout this paper.

\vspace{-0.75em}

\section*{Related Work}

Hilbert introduced a continuous space-filling curve from the unit interval to the unit square, known as the Hilbert curve \cite{hilbert1891}.
Its standard discrete form applies to square (2D) and cubic (3D) grids with equal power-of-two side lengths.

In a website application and scanned note, Tautenhahn \cite{lutanho2003} presented a 2D construction for arbitrary sizes based on combining
$3\times3$ Peano and $2\times2$ Hilbert block types.
While Tautenhahn's approach is 2D only and has non-uniform subdivision schemes dependent on side length parity,
it motivates an important endpoint-parity constraint that we make explicit.

Zhang, Kamata, and Ueshige \cite{zhang2006} propose a pseudo-Hilbert scan for arbitrary rectangles.
Compared to our construction, their method follows a more intricate case structure and does not
suggest a direct extension to 3D.

\vspace{-0.75em}

\section*{Parity Constraints on Corner Endpoints}
\label{sec:parity}

We use a parity argument to determine whether a Hamiltonian path is possible between two grid corner endpoints.
If we consider the standard checkerboard coloring of an $m\times n$ grid graph (or an $m\times n\times p$ grid graph in 3D),
then adjacent vertices always have opposite color.
Any path alternates colors at every step, so the parity of the endpoints is constrained by the total number of visited vertices.

\begin{lemma}[Color compatibility]
  \label{lemma:colorCompatibility}
  Let $G$ be a rectangular 2D grid graph with $(m \times n)$ vertices, or a rectangular 3D grid graph with $(m \times n \times p)$ vertices.
Fix a start corner, $s$, and an end corner, $t$.
If there exists a Hamiltonian path from $s$ to $t$, then $s$ and $t$ have the \emph{same} color when the number of vertices is odd,
and \emph{opposite} colors when the number of vertices is even.
\end{lemma}

\begin{proof}
Along any path, vertex colors alternate. A Hamiltonian path visits all $N$ vertices exactly once and therefore has $N-1$ steps.
If $N$ is odd, then $N-1$ is even and the endpoints must have the same color. If $N$ is even, then $N-1$ is odd and the endpoints must have opposite colors.
\end{proof}

We call a pair of endpoints \emph{color compatible} if they satisfy Lemma \ref{lemma:colorCompatibility}.
Color compatibility is necessary, in general, for Hamiltonicity. For the specific family of corner-to-corner subproblems produced by our construction,
it is also sufficient: a trivial linear zig-zag path connecting color compatible corners is always possible.


\begin{figure}[!htb]

  \begin{minipage}{0.55\textwidth}

    \begin{minipage}{1.0\textwidth}
      \centering
      \includegraphics[width=\linewidth]{gilbert2d_mainsubdiv.pdf}
      \caption{ Subdivision template for the 2D Gilbert curve. }
      \label{fig:main2dsubdiv}
    \end{minipage}

    \begin{minipage}{1.0\textwidth}
      \centering
      \includegraphics[width=0.65\linewidth]{gilbert2d_example_mainsubdiv.pdf}
      \caption{ Example curve with subdivision regions highlighted. }
      \label{fig:main2dsubdivExample}
    \end{minipage}

  \end{minipage}\hfill
  \begin{minipage}{0.45\textwidth}
    \centering
    \includegraphics[width=\linewidth]{gilbert3d_explode.pdf}
    \caption{ The main subdivision template for the 3D Gilbert curve. }
    \label{fig:gilbert3DJSplit}
  \end{minipage}

\end{figure}

\vspace{-0.75em}

\section*{Algorithms}
\label{sec:algorithms}

Algorithms~\ref{alg:gilbert2d} and~\ref{alg:gilbert3d} give pseudocode for the 2D and 3D Gilbert curve constructions.
%Both algorithms enumerate a Gilbert curve by repeated calls, returning a resolved point
%at every yield statement, then saving state so that it can continue when called again.

We represent an oriented grid-aligned rectangle or cuboid by an origin $p\in\mathbb{Z}^d$ and axis vectors
$\alpha,\beta,\gamma\in\mathbb{Z}^d$, $d \in \{2,3\}$, whose nonzero components encode the side lengths and directions.
The function $\delta(\cdot)$ returns the unit axis direction of its argument, and $\text{div}(\cdot,\cdot)$ denotes
integer division (rounding toward zero).
To make things more concise, a convenience function is used to help coerce values to be even in the the non-degenerate case:
$\Delta( \rho_2,\rho ) \overset{\mathrm{def}}{=}
%(\textit{ if } (|\rho _ 2| \equiv 1 \bmod{2} \text{ and } |\rho| > 2) \to \delta(\rho), \textit{ else } \to 0))$,
\delta(\rho)$ if ($|\rho_2|$ is odd and $|\rho|>2$), and $0$ otherwise.

The next sections go into detail about the 2D construction and its extension to 3D.


\section*{The 2D Gilbert Curve}
\label{sec:2d}

%Our 2D construction recursively partitions an $m\times n$ rectangle into a small number of axis-aligned sub-rectangles
The 2D Gilbert curve construction recursively partitions an $m\times n$ rectangle into a small number of axis-aligned sub-rectangles
and concatenates the subpaths into a single path. %Hamiltonian path.
Figure~\ref{fig:main2dsubdiv} shows the main 2D subdivision template and
Figure~\ref{fig:main2dsubdivExample} illustrates one recursion with subregions highlighted.

%The vectors $\alpha, \beta \in \mathbb{Z}^3$ provide generalized notions of width ($\alpha$), height ($\beta$),
%and depth ($\gamma$), so that we can transform rectangular or cuboid regions as necessary for the recursion.
The vectors $\alpha, \beta \in \mathbb{Z}^2$ provide generalized notions of width ($\alpha$) and height ($\beta$).
Each of $\alpha, \beta$ will only have one non-zero component and will be orthogonal to each other,
representing a snapshot of the current frame of reference.

When dividing into subregions, we choose integral side lengths and prefer an even local width-like length for the first region
(e.g. region $A$ in Figures \ref{fig:main2dsubdiv}, \ref{fig:main2dsubdivExample}) to satisfy the
parity constraint of Lemma \ref{lemma:colorCompatibility}. %Section~\ref{sec:parity}.
Each subregion is solved recursively and then stitched to its neighbors.
Figure \ref{fig:gilbert2d_bulk} shows a color coded subdivision template and Figure \ref{fig:gilbert2d_bulk_example} shows a color coded example with stitched endpoints highlighted.
%Figure~\ref{fig:production2d} summarizes the main parity-driven cases.

As a heuristic to keep subregions square like,
the parent region is bisected
if the aspect ratio of width to height exceeds a $3/2$ threshold.
After bisection, the algorithm recursively proceeds on each half as normal.
See Figure \ref{fig:eccentric_2d_alg} for the subdivision template and Figure \ref{fig:eccentric_2d_example_alg} for an example of the bisection subdivision.

If the width-like side length is odd with the height even, a diagonal move is forced as the endpoints won't be
color compatible anymore.
In such a case, there will be a single diagonal move in the resulting path and it will be guided to appear in the upper right
hand corner, as the $B$ region (Figure \ref{fig:main2dsubdiv}) will retain the odd width and even height.

In all other cases (even width, odd width and height), a Hamiltonian path is possible without diagonal moves.
When the side lengths are exact powers of two, the resulting curve is identical to the 2D Hilbert curve.
Algorithm~\ref{alg:gilbert2d} gives pseudo code for the 2D Gilbert curve.


\begin{minipage}[!htb]{0.43\linewidth}
\begin{algorithm}[H]
  \caption{Gilbert 2D}
  \label{alg:gilbert2d}
  \begin{algorithmic}

    \State \textit{\# $p, \alpha, \beta \in \mathbb{Z}^2$}
    \State \textbf{function }$\text{Gilbert2D}(p, \alpha, \beta)$
    %\Function{Gilbert2D}{$p, \alpha, \beta$}
      \vskip0.5em
      \State \hskip1.0em $\alpha_2, \beta_2  = \text{div}(\alpha, 2), \text{div}(\beta, 2)$
      \vskip0.5em
      %\If{ $(|\beta| \equiv 1)$ }
      \State \hskip1.0em \textbf{if }$(|\beta| \equiv 1)$
        \State \hskip2.0em \textbf{print} $p + i \cdot \delta(\alpha)$ \textbf{forall} $i \in |\alpha|$
      %\ElsIf{ $(|\alpha| \equiv 1)$ }
      \State \hskip1.0em \textbf{else if }$(|\alpha| \equiv 1)$
        \State \hskip2.0em \textbf{print} $p + i \cdot \delta(\beta)$ \textbf{forall} $i \in |\beta|$
      %\ElsIf{ $(2 |\alpha| > 3 |\beta|)$ }
      \vskip0.5em
      \State \hskip1.0em \textbf{else if }$(2 |\alpha| > 3 |\beta|)$
        %\If{ $(|\alpha_2| > 2)$ and \\ \hskip4.12em $(|\alpha_2| \bmod{2} \equiv 1)$ }
        %\State \hskip1.0em \textbf{if }$(|\alpha_2| > 2)$ and \\ \hskip3.4em $(|\alpha_2| \bmod{2} \equiv 1)$
        \vskip0.5em
        \State \hskip2.0em \textbf{if }$(|\alpha_2| > 2) \text{ and } (|\alpha_2| \bmod{2} \equiv 1)$
          \State \hskip3.0em $\alpha_2 \leftarrow \alpha_2 + \delta(\alpha)$
        %\EndIf
        \vskip0.5em
        \State \hskip2.0em Gilbert2D($p, \alpha_2, \beta$)
        \State \hskip2.0em Gilbert2D($p + \alpha_2, \alpha - \alpha_2, \beta$)
      %\Else
      \vskip0.5em
      \State \hskip1.0em \textbf{else }
        %\If{ $(|\beta_2| > 2)$ and \\ \hskip4.12em $(|\beta_2| \bmod{2} \equiv 1)$ }
        \vskip0.5em
        \State \hskip2.0em \textbf{if }$(|\beta_2| > 2) \text{ and } (|\beta_2| \bmod{2} \equiv 1)$
          \State \hskip3.0em $\beta_2 \leftarrow \beta_2 + \delta(\beta)$
        %\EndIf
        %\State Gilbert2D($p$, \\ \hskip7.35em $\beta_2, \alpha_2$)
        \vskip0.5em
        %\State \hskip2.0em Gilbert2D($p, \beta_2, \alpha_2$)
        \State \hskip2.0em Gilbert2D($p,$ \\ \hskip6.45em $\beta_2, \alpha_2$)
        \State \hskip2.0em Gilbert2D($p + \beta_2$, \\ \hskip6.45em $\alpha, (\beta - \beta_2)$)
        %\State \hskip2.0em Gilbert2D($p + (\alpha - \delta(\alpha)) \ +$ \\ \hskip6.45em $(\beta_2 - \delta(\beta))$, \\ \hskip6.2em $-\beta_2, -(\alpha - \alpha_2)$)
        \State \hskip2.0em Gilbert2D($p + (\alpha - \delta(\alpha))  + (\beta_2 - \delta(\beta))$, \\ \hskip6.2em $-\beta_2, -(\alpha - \alpha_2)$)
      \vskip0.5em
      %\EndIf
    %\EndFunction

%    \State \textit{\# $p, \alpha, \beta \in \mathbb{Z}^2$}
%    \Function{Gilbert2D}{$p$, $\alpha$, $\beta$}
%      \State $\alpha_2, \beta_2  = \text{div}(\alpha, 2), \text{div}(\beta, 2)$
%      \If{ $(|\beta| \equiv 1)$ }
%        \State \textbf{yield} $p + i \cdot \delta(\alpha)$ \textbf{forall} $i \in |\alpha|$
%      \ElsIf{ $(|\alpha| \equiv 1)$ }
%        \State \textbf{yield} $p + i \cdot \delta(\beta)$ \textbf{forall} $i \in |\beta|$
%      \ElsIf{ $(2 |\alpha| > 3 |\beta|)$ }
%        \If{ $(|\alpha_2| > 2)$ and \\ \hskip4.12em $(|\alpha_2| \bmod{2} \equiv 1)$ }
%          \State $\alpha_2 \leftarrow \alpha_2 + \delta(\alpha)$
%        \EndIf
%        \State \textbf{yield} Gilbert2D($p$, $\alpha_2$, $\beta$)
%        \State \textbf{yield} Gilbert2D($p + \alpha_2$, $\alpha - \alpha_2$, $\beta$)
%      \Else
%        \If{ $(|\beta_2| > 2)$ and \\ \hskip4.12em $(|\beta_2| \bmod{2} \equiv 1)$ }
%          \State $\beta_2 \leftarrow \beta_2 + \delta(\beta)$
%        \EndIf
%        \State \textbf{yield} Gilbert2D($p$, \\ \hskip9.75em $\beta_2$, $\alpha_2$)
%        \State \textbf{yield} Gilbert2D($p + \beta_2$, \\ \hskip9.75em $\alpha$, $(\beta - \beta_2)$)
%        \State \textbf{yield} Gilbert2D($p + (\alpha - \delta(\alpha)) +$ \\ \hskip9.75em $(\beta_2 - \delta(\beta))$, \\ \hskip9.75em $-\beta_2$, $-(\alpha - \alpha_2)$)
%      \EndIf
%    \EndFunction

  \end{algorithmic}
\end{algorithm}
\end{minipage}\hfill
\begin{minipage}[!htb]{0.43\linewidth}

  \begin{minipage}[ht]{0.49\linewidth}

    \centering
    \includegraphics[width=0.925\linewidth]{gilbert2d_eccentric_alpha1.pdf}
    \captionof{figure}{ }
    \label{fig:eccentric_2d_alg}

  \end{minipage}\hfill
  \begin{minipage}[ht]{0.47\linewidth}

    \centering
    \includegraphics[width=1.0\linewidth]{gilbert2d_example_eccentric.pdf}
    \captionof{figure}{ }
    \label{fig:eccentric_2d_example_alg}

  \end{minipage}

  \begin{minipage}[ht]{0.95\linewidth}

    \centering
    \includegraphics[width=0.925\linewidth]{gilbert2d_bulk.pdf}
    \captionof{figure}{ }
    \label{fig:gilbert2d_bulk}

  \end{minipage}

  \begin{minipage}[ht]{0.95\linewidth}

    \centering
    \includegraphics[width=1.0\linewidth]{gilbert2d_bulk_example.pdf}
    \captionof{figure}{ }
    \label{fig:gilbert2d_bulk_example}

  \end{minipage}

\end{minipage}

\vspace{-0.45em}

\section*{Extension to 3D}
\label{sec:3d}

In the basic case, a subdivision scheme is used to split the cuboid into five regions.
Two cube like regions are partitioned where the path starts and stops, and three oblong cuboid regions are partitioned
for the middle portion of the path.
Figure \ref{fig:gilbert3DJSplit} provides an exploded view of the main subdivision and where endpoints connect.
See also Figure \ref{fig:gilbert3d_alg_fig} (\textit{g}) for the template and Figure \ref{fig:gilbert3d_alg_fig} (\textit{h})
for a $10 \times 10 \times 10$ example, with each region color coded.

During subdivision, if the cuboid’s aspect ratio between width and either height or depth exceeds $3/2$,
a horizontal bisection is performed. Two additional anisotropic cases are handled analogously when the aspect
ratio between height and depth, or depth and height, exceeds a threshold of $4/3$.
These thresholds were chosen empirically; a principled derivation and further optimization are left for future work.
See Figure \ref{fig:gilbert3d_alg_fig} (\textit{a}-\textit{f}) for color coded templates and color coded examples.

Endpoints within a sub-divided region are kept on the exterior of the parent cuboid and joined after the resulting recursion
has completed.
In the case all side lengths are even, the subdivision will always choose an even length until the base case is
encountered, ensuring a Hamiltonian path and a realization free from diagonal moves.
As with the 2D Gilbert curve, when side dimensions are equal and exact powers of two, the resulting curve is identical to
the standard 3D Hilbert curve.

\begin{minipage}[ht]{0.5\linewidth}
\begin{algorithm}[H]
  \caption{Gilbert 3D}
  \label{alg:gilbert3d}
  \begin{algorithmic}

    \State \textit{\# $p, \alpha, \beta, \gamma \in \mathbb{Z}^3$}
    %\Function{Gilbert3D}{$p, \alpha, \beta, \gamma$}
    \State \textbf{function }$\text{Gilbert3D}(p, \alpha, \beta, \gamma)$

    \vskip0.5em
    \State \hskip1.0em $\alpha_2 \leftarrow \text{div}(\alpha,2) + \Delta(\alpha _ 2, \alpha)$
    \State \hskip1.0em $\beta_2 \leftarrow \text{div}(\beta,2) + \Delta(\beta _ 2, \beta)$
    \State \hskip1.0em $\gamma_2 \leftarrow \text{div}(\gamma,2) + \Delta(\gamma _ 2, \gamma)$

    \vskip0.75em

    \State \hskip1.0em \textbf{if }$(|\beta| \equiv 1)$ \textbf{ and } $(|\gamma| \equiv 1)$
      \State \hskip2.0em \textbf{print} $p + i \cdot \delta(\alpha)$ \textbf{forall} $i \in |\alpha|$
    \State \hskip1.0em \textbf{else if }$(|\alpha| \equiv 1)$ \textbf{ and } $(|\gamma| \equiv 1)$
      \State \hskip2.0em \textbf{print} $p + i \cdot \delta(\beta)$ \textbf{forall} $i \in |\beta|$
    \State \hskip1.0em \textbf{else if }$(|\alpha| \equiv 1)$ \textbf{ and } $(|\beta| \equiv 1)$
      \State \hskip2.0em \textbf{print} $p + i \cdot \delta(\gamma)$ \textbf{forall} $i \in |\gamma|$

    \vskip0.75em

    \State \hskip1.0em \textbf{else if }$(2 |\alpha|>3|\beta|) \text{ and } (2|\alpha|>3|\gamma|))$
      %\State \hskip2.0em  Gilbert3D($p,\alpha _ 2,\beta,\gamma$)
      \State \hskip2.0em  Gilbert3D($p,$ \\ \hskip6.75em $\alpha _ 2,\beta,\gamma$)
      %\State \hskip2.0em  Gilbert3D($p + \alpha _ 2,\alpha - \alpha _ 2,\beta,\gamma$)
      \State \hskip2.0em  Gilbert3D($p + \alpha _ 2,$ \\ \hskip6.75em $\alpha - \alpha _ 2,\beta,\gamma$)

    \vskip0.75em

    \State \hskip1.0em \textbf{else if }$(3 |\beta| > 4 |\gamma|)$
      %\State \hskip2.0em  Gilbert3D($p,\beta _ 2,\gamma,\alpha _ 2$)
      \State \hskip2.0em  Gilbert3D($p,$ \\ \hskip6.75em $\beta _ 2,\gamma,\alpha _ 2$)
      %\State \hskip2.0em  Gilbert3D($p + \beta _ 2,\alpha,\beta - \beta _ 2,\gamma$)
      \State \hskip2.0em  Gilbert3D($p + \beta _ 2,$ \\ \hskip6.75em $\alpha,\beta - \beta _ 2,\gamma$)
      %\State \hskip2.0em  Gilbert3D($p \ + $ \\ \hskip6.255em $(\alpha - \delta(\alpha)) +$ \\ \hskip6.25em $(\beta _ 2 - \delta(\beta))$, \\ \hskip6.0em $-\beta _ 2,\gamma,-(\alpha - \alpha _ 2)$)
      \State \hskip2.0em  Gilbert3D($p  + (\alpha - \delta(\alpha)) + (\beta _ 2 - \delta(\beta))$, \\ \hskip6.0em $-\beta _ 2,\gamma,-(\alpha - \alpha _ 2)$)

    \vskip0.75em

    \State \hskip1.0em \textbf{else if }$(3 |\gamma| > 4 |\beta|)$
      %\State \hskip2.0em  Gilbert3D($p,\gamma _ 2,\alpha _ 2,\beta$)
      \State \hskip2.0em  Gilbert3D($p,$ \\ \hskip6.75em $\gamma _ 2,\alpha _ 2,\beta$)
      %\State \hskip2.0em  Gilbert3D($p + \gamma _ 2,\alpha, \beta, \gamma - \gamma _ 2$)
      \State \hskip2.0em  Gilbert3D($p + \gamma _ 2,$ \\ \hskip6.75em $\alpha, \beta, \gamma - \gamma _ 2$)
      %\State \hskip2.0em  Gilbert3D($p \ +$ \\ \hskip6.25em $(\alpha - \delta(\alpha)) \ +$ \\ \hskip6.25em $(\gamma_2 - \delta(\gamma))$, \\ \hskip6.0em $-\gamma _ 2,-(\alpha-\alpha _ 2), \beta$)
      \State \hskip2.0em  Gilbert3D($p  + (\alpha - \delta(\alpha))  + (\gamma_2 - \delta(\gamma))$, \\ \hskip6.0em $-\gamma _ 2,-(\alpha-\alpha _ 2), \beta$)

    \vskip0.75em

    \State \hskip1.0em \textbf{else }
      %\State \hskip2.0em  Gilbert3D($p, \beta _ 2, \gamma _ 2, \alpha _ 2$)
      \State \hskip2.0em  Gilbert3D($p,$ \\ \hskip6.75em $\beta _ 2, \gamma _ 2, \alpha _ 2$)
      %\State \hskip2.0em  Gilbert3D($p + \beta _ 2,\gamma,\alpha _ 2,(\beta - \beta _ 2)$)
      \State \hskip2.0em  Gilbert3D($p + \beta _ 2,$, \\ \hskip6.75em $\gamma,\alpha _ 2,(\beta - \beta _ 2)$)
      %\State \hskip2.0em  Gilbert3D($p \ +$ \\ \hskip6.25em $(\beta _ 2 - \delta(\beta)) \ +$ \\ \hskip6.25em $(\gamma - \delta(\gamma))$, \\ \hskip6.5em $\alpha,-\beta_2,-(\gamma-\gamma_2)$)
      \State \hskip2.0em  Gilbert3D($p  + (\beta _ 2 - \delta(\beta))  + (\gamma - \delta(\gamma))$, \\ \hskip6.5em $\alpha,-\beta_2,-(\gamma-\gamma_2)$)
      %\State \hskip2.0em  Gilbert3D($p \ +$ \\ \hskip6.25em $(\alpha - \delta(\alpha)) \ + \ $ \\ \hskip6.5em $\beta _ 2 + (\gamma - \delta(\gamma))$, \\ \hskip6.00em $-\gamma,-(\alpha - \alpha_2),(\beta-\beta_2)$)
      \State \hskip2.0em  Gilbert3D($p + (\alpha - \delta(\alpha)) + \beta _ 2 + (\gamma - \delta(\gamma))$, \\ \hskip6.00em $-\gamma,-(\alpha - \alpha_2),(\beta-\beta_2)$)
      %\State \hskip2.0em  Gilbert3D($p \ +$ \\ \hskip6.25em $(\alpha - \delta(\alpha)) \ +$ \\ \hskip6.25em $(\beta _ 2 - \delta(\beta))$, \\ \hskip6.0em $-\beta _ 2,\gamma _ 2,-(\alpha - \alpha_2)$)
      \State \hskip2.0em  Gilbert3D($p + (\alpha - \delta(\alpha))  + (\beta _ 2 - \delta(\beta))$, \\ \hskip6.0em $-\beta _ 2,\gamma _ 2,-(\alpha - \alpha_2)$)

    \vskip1.0em

%    \EndFunction

%    \State \textit{\# $p, \alpha, \beta, \gamma \in \mathbb{Z}^3$}
%    \Function{Gilbert3D}{$p$, $\alpha$, $\beta$, $\gamma$}
%
%    \vskip0.5em
%    \State $\alpha_2 \leftarrow \text{div}(\alpha,2) + \Delta(\alpha _ 2, \alpha)$
%    \State $\beta_2 \leftarrow \text{div}(\beta,2) + \Delta(\beta _ 2, \beta)$
%    \State $\gamma_2 \leftarrow \text{div}(\gamma,2) + \Delta(\gamma _ 2, \gamma)$
%
%    \vskip0.5em
%
%    \State \textbf{if }$(|\beta| \equiv 1)$ \textbf{ and } $(|\gamma| \equiv 1)$
%      \State \hskip1.0em \textbf{yield} $p + i \cdot \delta(\alpha)$ \textbf{forall} $i \in |\alpha|$
%    \State \textbf{else if }$(|\alpha| \equiv 1)$ \textbf{ and } $(|\gamma| \equiv 1)$
%      \State \hskip1.0em \textbf{yield} $p + i \cdot \delta(\beta)$ \textbf{forall} $i \in |\beta|$
%    \State \textbf{else if }$(|\alpha| \equiv 1)$ \textbf{ and } $(|\beta| \equiv 1)$
%      \State \hskip1.0em \textbf{yield} $p + i \cdot \delta(\gamma)$ \textbf{forall} $i \in |\gamma|$
%
%    \vskip0.5em
%
%    \State \textbf{else if }$(2 |\alpha|>3|\beta|) \text{ and } (2|\alpha|>3|\gamma|))$
%      \State \hskip1.0em \textbf{yield} Gilbert3D($p$,$\alpha _ 2$,$\beta$,$\gamma$)
%      \State \hskip1.0em \textbf{yield} Gilbert3D($p + \alpha _ 2$,$\alpha - \alpha _ 2$,$\beta$,$\gamma$)
%    \State \textbf{else if }$(3 |\beta| > 4 |\gamma|)$
%      \State \hskip1.0em \textbf{yield} Gilbert3D($p$,$\beta _ 2$,$\gamma$,$\alpha _ 2$)
%      \State \hskip1.0em \textbf{yield} Gilbert3D($p + \beta _ 2$,$\alpha$,$\beta - \beta _ 2$,$\gamma$)
%      \State \hskip1.0em \textbf{yield} Gilbert3D($p + $ \\ \hskip9.125em $(\alpha - \delta(\alpha)) +$ \\ \hskip9.125em $(\beta _ 2 - \delta(\beta))$, \\ \hskip9.125em $-\beta _ 2$,$\gamma$,$-(\alpha - \alpha _ 2)$)
%    \State \textbf{else if }$(3 |\gamma| > 4 |\beta|)$
%      \State \hskip1.0em \textbf{yield} Gilbert3D($p$,$\gamma _ 2$,$\alpha _ 2$,$\beta$)
%      \State \hskip1.0em \textbf{yield} Gilbert3D($p + \gamma _ 2$,$\alpha$, $\beta$, $\gamma - \gamma _ 2$)
%      \State \hskip1.0em \textbf{yield} Gilbert3D($p +$ \\ \hskip9.125em $(\alpha - \delta(\alpha))$ \\ \hskip9.125em $(\gamma_2 - \delta(\gamma))$, \\ \hskip9.5em $-\gamma _ 2$,$-(\alpha-\alpha _ 2)$, $\beta$)
%
%    \State \textbf{else }
%      \State \hskip1.0em \textbf{yield} Gilbert3D($p$,$\beta _ 2$,$\gamma _ 2$,$\alpha _ 2$)
%      \State \hskip1.0em \textbf{yield} Gilbert3D($p + \beta _ 2$,$\gamma$,$\alpha _ 2$,$(\beta - \beta _ 2)$)
%      \State \hskip1.0em \textbf{yield} Gilbert3D($p +$ \\ \hskip9.125em $(\beta _ 2 - \delta(\beta)) +$ \\ \hskip9.125em $(\gamma - \delta(\gamma))$, \\ \hskip9.5em $\alpha$,$-\beta_2$,$-(\gamma-\gamma_2)$)
%      \State \hskip1.0em \textbf{yield} Gilbert3D($p +$ \\ \hskip9.125em $(\alpha _ 2 - \delta(\alpha)) +$ \\ \hskip9.125em $\beta _ 2 + (\gamma - \delta(\gamma))$, \\ \hskip9.5em $-\gamma$,$-(\alpha - \alpha_2)$,$(\beta-\beta_2)$)
%    \State \hskip1.0em \textbf{yield} Gilbert3D($p +$ \\ \hskip9.125em $(\alpha - \delta(\alpha)) +$ \\ \hskip9.125em $(\beta _ 2 - \delta(\beta))$, \\ \hskip9.5em $-\beta _ 2$,$\gamma _ 2$,$-(\alpha - \alpha_2)$)
%
%    \EndFunction

  \end{algorithmic}
\end{algorithm}
\end{minipage}\hfill
\begin{minipage}[ht]{0.5\linewidth}

    \centering
    \includegraphics[width=\linewidth]{gilbert3d_alg_fig3.pdf}
    \captionof{figure}{ }
    \label{fig:gilbert3d_alg_fig}

\end{minipage}

If any of the subdivided regions has endpoints that lay in a direction with
odd side length, a diagonal move will appear.
As such, with the subdivision scheme presented, the number of diagonal moves in the resulting curve isn't limited to one.


\section*{Summary and Conclusions}
\label{sec:outlook}

%The Gilbert curve provides locality-preserving traversals in rectangular grids of arbitrary dimensions.
The Gilbert curve provides locality-preserving traversals in rectangular grids of arbitrary side lengths.
For even side lengths, the construction yields uniform, defect-free Hamiltonian paths in both two and three dimensions,
providing a natural generalization of classical discrete Hilbert curves beyond power-of-two domains.
For grids with odd dimensions, the resulting curves may contain diagonal steps, but still retain strong
locality properties and remain suitable for practical use in applications such as cache-coherent traversal and image or volume compression.

While the anisotropic split thresholds are empirically chosen, they work well in practice; future work could formalize optimization criteria for selecting subdivision templates.
In 3D, relaxing subproblem endpoint constraints may allow constructions that further reduce
the occurrence of forced diagonal steps when one or more side lengths are odd.

An open source implementation for the Gilbert curve in 2D and 3D has been developed and can be downloaded from its
repository\footnote{ \label{gilbert-url} \url{https://github.com/jakubcerveny/gilbert} }.

\section*{Acknowledgments}

AI tools were used in assisting the language and readability of this paper.

{\setlength{\baselineskip}{13pt} % tighten line spacing for bibliography
\raggedright        % no right justification for References
\begin{thebibliography}{99}

  \bibitem{anders2009} S. Anders. ``Visualization of genomic data with the Hilbert curve.'' \textit{Bioinformatics}, vol.~25, no.~10, pp. ~ 1231-1235, 2009.  \url{https://doi.org/10.1093/bioinformatics/btp152}

  \bibitem{bohm2021} C. B{\"o}hm, M. Perdacher, and C. Plant ``A Novel Hilbert Curve for Cache-Locality Preserving Loops.'' \textit{IEEE Transactions on Big Data}, vol.~7, 2021, pp.~241-254

  \bibitem{cortesi2011} A. Cortesi. \textit{Visualizing binaries with space-filling curves}. 2011. \url{https://corte.si/posts/visualisation/binvis/}.

  %\bibitem{sagan1994} H. Sagan. \textit{Space-Filling Curves}, 1994 ed. Springer, 1994.

  \bibitem{hilbert1891} D. Hilbert. ``Ueber die stetige Abbildung einer Linie auf ein Fl{\"a}chenst{\"u}ck.'', \textit{Mathematische Annalen}, vol.~38, 1891, pp.~459-460

  %\bibitem{lee2025} S. Lee, K. Ma. ``HINTs: Sensemaking on large collections of documents with {\bf H}ypergraph vizualization and {\bf INT}elligent agents.'' \textit{IEEE Transactions on Visualization and Computer Graphics}, vol.~31, no.~9, 2025, pp.~5532-5546

  \bibitem{moon2001} B. Moon, H. Jagadish, C. Faloutsos, and J. Saltz. ``Analysis of the clustering properties of the Hilbert space-filling curve.'' \textit{IEEE Transactions on Knowledge and Data Engineering}, vol.~13, no.~1, 2001, pp.~124-141.

  \bibitem{munroe2006} R. Munroe. \textit{Map of the Internet}. xkcd. 2006. \url{https://xkcd.com/195/}

  %\bibitem{rong2021} Y. Rong, X. Zhang, and J. Lin. ``

  \bibitem{lutanho2003} L. Tautenhahn. \textit{Draw a space-filling curve of arbitrary size}, 2003. \url{https://lutanho.net/pic2html/draw_sfc.html}.

  \bibitem{zhang2006} J. Zhang, S. Kamata, and Y. Ueshige. ``A Pseudo-Hilbert Scan Algorithm for Arbitrarily-Sized Rectangle Region.'' \textit{Advances in Machine Vision, Image Processing, and Pattern Analysis}, Berlin, Heidelberg, 2006, pp. 290-299.
\end{thebibliography}
} % end setlength, raggedright

%\begin{figure}[!htb]
%  \centering
%  %{\transparent{0.4}\includegraphics[width=\linewidth]{gilbert_bd50x10.pdf}}
%  \includegraphics[width=0.9\linewidth]{gilbert_bd50x10.pdf}
%  %\includegraphics[width=0.8\linewidth]{gilbert_fun_50x6.pdf}
%  \label{fig:footerGilbert}
%\end{figure}


\end{document}


