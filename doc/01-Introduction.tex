\section{Introduction}

\subsection{Overview}

We present a generalized Hilbert curve for 2D and 3D curves
that works on arbitrary rectangular regions.
We call our realization of the generalized Hilbert curve a \textit{Gilbert curve}.

An overview of the benefits of the Gilbert curve are that it is:

\begin{itemize}
  \item Valid on arbitrary rectangular regions
  \item Equivalent to the Hilbert curve when dimensions are exact powers of 2
  \item Efficient at random access lookup
  \item Algorithmically conceptually straight forward
  \item Able to realize curves without notches unnecessarily and limits the resulting curve
        to a single notch when forced
  \item Similar in measures of locality to the Hilbert curve
\end{itemize}

Some drawbacks are that:

\begin{itemize}
  \item Our implementation involves a recursive solution for random access lookup
        which might be too slow for applications that require better than $O( \lg N)$
        runtime and memory usage
  \item Might not adequately capture some features of a Hilbert curve
\end{itemize}

Further, we show:

\begin{itemize}
  \item Measures of locality are preserved (Section X)
  \item Trivial extensions to create generalized Moore curves (\textit{Gore curve}, Section X)
  \item Other metrics to quantify qualities of the curve and their comparison to some other
        space filling curves
\end{itemize}


Generalizations of the Hilbert curve to non power of two rectangular cuboid regions
has been explored before but are overly complicated algorithmically, create unbalanced
curves and often don't generalize well to 3D.
Our realization focuses on algorithmic conceptual simplicity, provides balanced
resulting curves and works in both 2D and 3D.

Space filling curves are a specialization of a more general Hamiltonian path,
but have a more stringent requirement of local connectivity.
The local connectivity, or locality, preserves a measure of nearness, where
points from the source unit line are near in the embedded space.

The local connectivity requirements
preclude things like \textit{zig-zag} Hamiltonian
paths, as nearby points in the embedded dimension can be far from the origin line.

% FIGURE DESCRIBING LOCALITY, LOCAL CONNECTIVITY


The Gilbert curve algorithm works by choosing sub rectangular cuboid regions, or blocks,
to recursively find a connecting path.
During the course of subdivision, if the cuboid block regions stray too far from being
cube like, a simpler subdivision is done to try and bring the subdivision closer to being
cube like.
The subdivision is done until a base case is reached and the lowest unit of the curve
can be realized.

For a specified path start and end position with certain side dimensions, a Hamiltonian
path is not always possible.
In such a case, the Gilbert curve will introduce a single diagonal path move, called a \textit{notch}.


\subsection{Definitions}

We concern ourselves with a space filling curve, $\omega$, through a rectangular region $( N _ x, N _ y, N _ z)$,  $N = (N _ x \cdot N _ y \cdot N _ z)$:

%  = & \sqrt{ (x _ {k-1} - x _ k)^2 + (y _ {k-1} - y _ k)^2 + (z _ {k-1} - z _ k)^2 } = 1 \\

$$
\begin{array}{rl}
  k \in & \{ 0 \dots (N-1) \}, \\
  & x _ k, y _ k, z _ k \in \mathbb{N} \\
  \omega _ k = & ( x _ k, y _ k, z _ k ) \\
  \omega = & ( \omega _ 0, \omega _ 1, \dots, \omega _ {N-1} ) \\
  k>1  & \to | \omega _ {k - 1} - \omega _ {k} | = 1 \\
  \forall i,j \in & \{ 0 \dots (N-1) \}, i \ne j  \\
  \to & \omega _ i \ne \omega _ j
\end{array}
$$


