\section{Introduction}

\subsection{Overview}

We present the \textit{Gilbert curve}, a generalized Hilbert curve for 2D and 3D,
that works on arbitrary rectangular regions.
%We call our realization of the generalized Hilbert curve a \textit{Gilbert curve}.

An overview of the benefits of the Gilbert curve are that it is:

\begin{itemize}
  \item Valid on arbitrary rectangular regions
  \item Equivalent to the Hilbert curve when dimensions are exact powers of 2
  \item Efficient at random access lookups ($O(\lg N)$)
  \item A conceptually straight forward algorithm
  \item Able to realize curves without notches unnecessarily and limits the resulting curve
        to a single notch when forced
  \item Similar in measures of locality to the Hilbert curve
\end{itemize}

Some drawbacks are that:

\begin{itemize}
  \item Our implementation is recursive,
        which may be undesirable for some applications requiring better than $O( \lg N)$
        memory usage or better constant bounds for runtime performance
  \item Might not adequately capture some features of a Hilbert curve
\end{itemize}

Further, we show:

\begin{itemize}
  \item Measures of locality are preserved (Section X)
  \item Trivial extensions to create generalized Moore curves (Section X)
  \item Comparison metrics to other space filling curves (Section X)
\end{itemize}


Generalizations of the Hilbert curve to non power of two rectangular cuboid regions
have been explored before but are overly complicated algorithmically, create unbalanced
curves and often don't generalize well to 3D.
Our realization focuses on a conceptually simple algorithm which creates pleasing
resulting curves and works in both 2D and 3D.

Space filling curves are a specialization of a more general Hamiltonian path,
but have a more stringent requirement of local connectivity.
The local connectivity, or locality, preserves a measure of nearness, where
points from the source unit line remain close in the embedded space.

The local connectivity requirements
preclude things like \textit{zig-zag} Hamiltonian
paths, as nearby points in the embedded dimension can be far from the origin line.

% FIGURE DESCRIBING LOCALITY, LOCAL CONNECTIVITY


The Gilbert curve algorithm works by choosing sub rectangular cuboid regions, or blocks,
to recursively find a connecting path.
If one extent of the cuboid becomes too large or small relative to the other lengths,
a special subdivision is performed to try and make further subdivisions more cube-like.
Subdivisions are performed until a base case is reached and the lowest unit of the curve
can be realized.

A Hamiltonian path is not always realizable for certain side lengths and endpoint constraints.
In such a case, the Gilbert curve will introduce a single diagonal path move, called a \textit{notch}.


\subsection{Definitions}

We concern ourselves with a space filling curve, $\omega = (\omega_0, \omega_1, \dots, \omega_{N-1})$,
through a rectangular region $( N _ x, N _ y, N _ z)$,  $N = (N _ x \cdot N _ y \cdot N _ z)$.

%  = & \sqrt{ (x _ {k-1} - x _ k)^2 + (y _ {k-1} - y _ k)^2 + (z _ {k-1} - z _ k)^2 } = 1 \\

$$
\begin{array}{rl}
  k \in & \{ 0 \dots (N-1) \}, \\
  & x _ k, y _ k, z _ k \in \mathbb{N} \\
  \omega _ k = & ( x _ k, y _ k, z _ k ) \\
  \omega = & ( \omega _ 0, \omega _ 1, \dots, \omega _ {N-1} ) \\
  k>1  & \to | \omega _ {k - 1} - \omega _ {k} | = 1 \\
  \forall i,j \in & \{ 0 \dots (N-1) \}, i \ne j  \\
  \to & \omega _ i \ne \omega _ j
\end{array}
$$

For convenience, we use curves in 3D with the understanding that the 2D case is a specialization
when $N_z = 1$ and drop the third dimension when convenient and context is clear.

A \textit{Hamiltonian path} is a path through a graph such that every vertex is visited exactly once.
In the context of this paper, we concern ourselves with Hamiltonian paths through the implied
graph from integral points in 3D restricted to a cuboid region $(N_x, N_y, N_z)$.
Each point in the cuboid represents a vertex in the implied graph, with vertices joined in the graph
if points are (Euclidean) distance one away from a neighboring point restricted to the cuboid.

We also concern ourselves with Hamiltonian paths that specify a \textit{start} and \textit{end} point within
a cuboid region.
%If a Hamiltonian path is possible within a cuboid region, the region will be called \textit{feasible}.


\textbf{TODO}

\begin{itemize}
  \item Define locality/local connectivity
  \item Define space filling
  \item Define admissible/feasible for Hamiltonian path?
  \item Define ``eccentric'' (oblong?)
\end{itemize}


