\section{Introduction}

\subsection{Overview}

We present the \textit{Gilbert curve}, a generalized Hilbert curve for 2D and 3D,
that works on arbitrary rectangular regions.
%We call our realization of the generalized Hilbert curve a \textit{Gilbert curve}.

An overview of the benefits of the Gilbert curve are that it is:

\begin{itemize}
  \item Valid on arbitrary rectangular regions
  \item Equivalent to the Hilbert curve when dimensions are exact powers of 2
  \item Efficient at random access lookups ($O(\lg N)$)
  \item A conceptually straight forward algorithm
  \item Able to realize curves without notches unnecessarily and limits the resulting curve
        to a single notch when forced
  \item Similar in measures of locality to the Hilbert curve
\end{itemize}

Further, we show:

\begin{itemize}
  \item Measures of locality are preserved (Section X)
  \item Comparison metrics to other space filling curves (Section X)
\end{itemize}

Some drawbacks are that:

\begin{itemize}
  \item Our implementation is recursive,
        which may be undesirable for some applications requiring better than $O( \lg N)$
        memory usage or better constant bounds for runtime performance
  \item Might not adequately capture some features of a Hilbert curve
\end{itemize}


Generalizations of the Hilbert curve to non power of two rectangular cuboid regions
have been explored before but are overly complicated algorithmically, create unbalanced
curves and often don't generalize well to 3D.
Our realization focuses on a conceptually simple algorithm which creates pleasing
resulting curves and works in both 2D and 3D.

Space filling curves are a specialization of a more general Hamiltonian path,
but have a more stringent requirement of local connectivity.
The local connectivity, or locality, preserves a measure of nearness, where
points from the source unit line remain close in the embedded space.

% this section might be unncecessary
%
The local connectivity requirements
preclude things like \textit{zig-zag} Hamiltonian
paths, paths that run straight from one side to another before turning around,
as nearby points in the embedded dimension can be far from the origin line.

% FIGURE DESCRIBING LOCALITY, LOCAL CONNECTIVITY


The Gilbert curve algorithm works by choosing sub rectangular cuboid regions
to recursively find a connecting path.
If one extent of the cuboid becomes too large or small relative to the other lengths,
a special subdivision is performed to try and make further subdivisions more cube-like.
Subdivisions are performed until a base case is reached and the lowest unit of the curve
can be realized.

A Hamiltonian path is not always realizable for certain side lengths and endpoint constraints.
In such a case, the Gilbert curve will introduce a single diagonal path move, called a \textit{notch}.


\subsection{Definitions}

We concern ourselves with a curves
that lie on the \textit{hypercubic lattice} $\mathbb{Z}^3$ graph,
where vertices are the integral points in dimension $3$ with edges joined between them when they
are unit distance away.

We can represent the curve as, $\omega = (\omega_0, \omega_1, \dots, \omega_{N-1})$, where $w_i \in \mathbb{Z}^3$.
A \textit{self avoiding curve} $\omega$, is one that does not visit a previously vertex ($i \ne j \to \omega_i \ne \omega_j$).
A \textit{Hamiltonian path} is a valid path through a graph that visits every vertex exactly once, starting
and ending at pre-specified points.

A curve restricted to a rectangular region $( N _ x, N _ y, N _ z)$,  $N = (N _ x \cdot N _ y \cdot N _ z)$
is said to have a Hamiltonian path \textit{feasible} if there exists a Hamiltonian path, with endpoints specified,
that hits every vertex within the restricted the rectangular region exactly once, starts at the specified start
point, $\omega_0$ and ends at the specified endpoint $\omega_{N-1}$.
A region that cannot have a Hamiltonian path, with the specified endpoints, is said to be \textit{infeasible}.

A consequence of the Hahn-Mazurkiewicz \cite{sagan_1994} one can map the unit interval to a three dimensional
cube if and only if the mapped to space is compact, connected and locally connected.
We concern ourselves with the concept of \textit{local connectivity} with attention to approximations
to finite realizations of curves on the 2D and 3D lattice.

We only provide some intuitive explanation for local connectivity here.
For our purposes, \textit{local connectivity} denotes the idea that if $|i-j|$ is small, then $|w_i - w_j|$
should also be small by some measure.
That is, nearness in the line domain should relate to nearness in the three dimensional range.
We use metrics to get a handle on local connectivity in a discrete and finite setting and leave
their definition and usage until in Section 5.

The concept of local connectivity gives is one of the key characteristics of a space filling curve
and differentiates it from other curves that can fill rectangular regions but don't do so in a
locally connected way \footnote{For example, the \textit{zig-zag} curve that creates a straight line until
it hits the bounds of the region to turn around and increment in the other dimensional lengths to fill 
a rectangular region.}.
The local connectivity, or \textit{locality}, is the feature that allows features in one dimension to
remain clustered in higher dimensions.

We refer the reader to Sagan 1994 \cite{sagan_1994} for a more thorough introduction to local connectivity.


%For simplicity, we focus on three dimensions with the understanding that
%the two dimensional case can be reduced to a three dimensional case with at least one unit side length.




\textbf{TODO}

\begin{itemize}
  \item Define locality/local connectivity
  \item Define space filling
  \item Define admissible/feasible for Hamiltonian path?
  \item Define ``eccentric'' (oblong?)
\end{itemize}


