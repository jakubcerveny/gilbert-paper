
\clearpage

\section{Appendix}


\subsection{Defect}

Call the \textit{defect} a function $\lambda _ d : \mathbb{N}^d \mapsto \mathbb{N}$:

$$
\begin{array}{l}
\lambda _ 2 (w,h) = \frac{ w \cdot h }{ \text{min}(w,h)^2 } \\
\lambda _ 3 (w,h,d) = \frac{ w \cdot h \cdot d }{ \text{min}(w,h,d)^3 }
\end{array}
$$

If there is a disjoint subdivision of a volume $V_0$ to $V_1  = ( V _ {0,0}, V _ {0,1}, \dots, V _ {0,m-1} )$,
$V _ 0  = \cup _ {k} V _ {0,k}$,
define the \textit{average defect} of the subdivided volume to be:

$$
\begin{array}{l}
  \lambda _ {s} ( V _ 1 ) = \sum _ {k} \frac{ \text{Vol}(V _ {0,k}) }{ \text{Vol}( V _ 0 ) } \cdot \lambda( V _ {0,k} )
\end{array}
$$

This weights the defect of each subdivided cuboid by its proportional volume.

The defect gives us a coarse idea of how lopsided or \textit{eccentric} a cuboid region is.
If the defect is too high, we might want to split the larger sides while keeping the smaller sides the same size.

\subsection{Eccentric Split Threshold}

Calculations in this section will justify what threshold value to pick of when to choose an
eccentric split over a J-split.
Our concern is to find a simple ratio of when each of $w$, $h$ or $d$ are considered
"small enough" or "large enough", relative to the other dimensions, to split on.

An enumeration of what conditions lead to an eccentric split are as follows:

$$
\begin{array}{l}
  w >> h \sim d (1) \\
  h >> w \sim d (2) \\
  d >> w \sim h (3) \\
  h << w \sim d (4) \\
  w << h \sim d (5) \\
  d << w \sim h (6) \\
\end{array}
$$


A representation of the eccentric splits are enumerated in figure \ref{fig:eccentric}.
The eccentric split differs from the J-split as it's only splitting in one or two
dimensions, not the full three that the J-split would be doing.

For each of the six cases, we want to know what relative difference in sizes should be used
to determine when an eccentric split is used and how to subdivide the cuboids so as to make progress.

Specifically, we want to find the ratio, $\sigma$ or $\eta$, of when one dimension is proportionally larger or smaller,
respectively, than the others and the ratio, $\rho$, of where to choose the split point of subdivision.
For simplicity, we might want to subdivide at the half way point ($\rho = \frac{1}{2}$) but as we
will see, this might give lopsided sub-cuboid regions and using a better split point is desirable.

In what follows, our goal is to choose a subdivision that will reduce the average defect.
We assume that the start and end of the path lie in the $w$ dimension with the local start point at $(0,0,0)$
and endpoint at $(w-1,0,0)$.

Because we want to avoid adding unnecessary notches, we are forced split in the $w$ axis.
We commit to always splitting the $w$ axis in two.
If we want to subdivide into three cuboid regions, we also need to pick the split point in the other dimension another dimension

\subsection{$w >> h \sim d$}

Take $w = \sigma s$ and $h = d = s$.
We commit to splitting the width dimension in half, subdividing the original volume into two equal volumes.

The defect of the original volume is $\lambda(w,h,d) = \sigma$.

If $w > 2h = 2s$, then $\text{min}(\frac{w}{2},h,d) = s$ and we have an average defect $\lambda_s(V_1) = \frac{\sigma}{2}$.
Intuitively, this is validation that if the width is more than twice the length of the depth and height, we make progress
if we split the width in half and recursively process each sub-cuboid.

If $w < 2h$ but $\text{min}(\frac{w}{2},h,d) = \frac{w}{2}$,
the average defect $\lambda_s(V_1) = 2 (\frac{1}{2}) \frac{ \frac{\sigma}{2} \cdot s^3 }{ (\frac{\sigma}{2} s)^3 } = \frac{4}{\sigma^2}$.
We want to reduce the average defect relative to the original defect, so:

$$
\begin{array}{ll}
  & \lambda_s (V_1) < \lambda(V_0) \\
  \to & \frac{4}{\sigma^2} < \sigma \\
  \to & \sigma > 4^{\frac{1}{3}} \\
  \to & \frac{5}{3} > \sigma > 4^{\frac{1}{3}} = 1.58740\dots \\
\end{array}
$$

We've chosen the constant $\frac{5}{3}$ as a simple fraction above $4^{\frac{1}{3}}$ to know when
to split $w$.

For this case, $\sigma = \frac{5}{3}$ and $\rho = \frac{1}{2}$.
That is, we split the $w$ axis by half when the width axis exceed $\frac{5}{3}$ of both the depth, $d$, and height, $h$, dimension.

\subsection{$h >> w \sim d$}


