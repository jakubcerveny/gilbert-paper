
%\clearpage

\section{Appendix}


\subsection{Defect} \label{appendix:defect}

Call the \textit{defect} a function $\lambda _ d : \mathbb{N}^d \mapsto \mathbb{N}$:

$$
\begin{array}{l}
\lambda _ 2 (|\alpha|,|\beta|) = \frac{ |\alpha| \cdot |\beta| }{ \text{min}(|\alpha|,|\beta|)^2 } \\
\lambda _ 3 (|\alpha|,|\beta|,|\gamma|) = \frac{ |\alpha| \cdot |\beta| \cdot |\gamma| }{ \text{min}(|\alpha|,|\beta|,|\gamma|)^3 }
\end{array}
$$

If there is a disjoint subdivision of a volume $V_0$ to $V_1  = ( V _ {0,0}, V _ {0,1}, \dots, V _ {0,m-1} )$,
$V _ 0  = \cup _ {k} V _ {0,k}$,
define the \textit{average defect} of the subdivided volume to be:

$$
\begin{array}{l}
  \lambda _ {s} ( V _ 1 ) = \sum _ {k} \frac{ \text{Vol}(V _ {0,k}) }{ \text{Vol}( V _ 0 ) } \cdot \lambda( V _ {0,k} )
\end{array}
$$

This weights the defect of each subdivided cuboid by its proportional volume.

The defect gives us a coarse idea of how lopsided or \textit{eccentric} a cuboid region is.
If the defect is too high, we might want to split the larger sides while keeping the smaller sides the same size.

\subsection{2D Eccentric Split Threshold}

For the 2D Gilbert curve,
if the $\alpha$ side length is significantly longer than the $\beta$ side length,
we want to subdivide the rectangle into two nearly equal regions and recursively find
a Gilbert curve in each region.
We will justify the constant $(3/2)$ as the ratio threshold to split on using an argument
given by L. Tautenhahn \footnote{ through personal communication }.
That is, when $(|\alpha|/|\beta| > 3/2)$, we split the rectangle into roughly equal parts.

The defect of a rectangle of side length $|\alpha|$ and $|\beta|$ is, with $|\alpha| > |\beta|$:

$$
\begin{array}{ll}
  \lambda_2(|\alpha|, |\beta|) & = |\alpha| \cdot |\beta| / |\beta|^2 \\
   & = |\alpha| / |\beta|
\end{array}
$$

After a subdivision, if we assume $|\alpha| < 2 |\beta|$, the defect is:

$$
\begin{array}{ll}
  \lambda_2(|\alpha|/2, |\beta|) & = \frac{|\alpha|}{2} \cdot |\beta| / (\frac{|\alpha|}{2})^2 \\
   & = 2 |\beta| / |\alpha|
\end{array}
$$

We're looking for the condition when there's a defect reduction, so

$$
\begin{array}{ll}
    & \lambda_2(|\alpha|/2, |\beta|)  < \lambda_2(|\alpha|, |\beta|) \\
\to & 2 |\beta| / |\alpha| < |\alpha| / |\beta| \\
  \to & \sqrt{2} < |\alpha| / |\beta| \\
\end{array}
$$

Since $\sqrt{2} \approx 1.4142 < (3/2)$, if we choose $|\alpha| / |\beta| > (3/2)$ we can
be assured a defect reduction.


\subsection{3D Eccentric Split Threshold}

Calculations in this section will justify what threshold value to pick of when to choose a 3D
eccentric split over a J-split.
Our concern is to find a simple ratio of when each of $\alpha$, $\beta$ or $\gamma$ are considered
"small enough" or "large enough", relative to the other dimensions, to split on.

An enumeration of what conditions lead to an eccentric split are as follows:

$$
\begin{array}{ll}
  (1) & |\alpha| >> |\beta| \sim |\gamma| \\
  (2) & |\beta| >> |\alpha| \sim |\gamma| \\
  (3) & |\gamma| >> |\alpha| \sim |\beta| \\
  (4) & |\beta| << |\alpha| \sim |\gamma| \\
  (5) & |\alpha| << |\beta| \sim |\gamma| \\
  (6) & |\gamma| << |\alpha| \sim |\beta| \\
\end{array}
$$

A representation of the eccentric splits are enumerated in Figure \ref{fig:gilbert3DEccentricCase}.
The eccentric $S$-split split differs from the $J$-split as it's only partitioning the cuboid along
one or two dimensions instead of three.

For each of the six cases, we want to know what relative difference in sizes should be used
to determine when an eccentric split should be used and how to subdivide the cuboids so as to make
the resulting subdivided cuboids more cube-like.

Specifically, we want to find the ratio, $\sigma$, of when one dimension is proportionally larger or smaller,
respectively, than the others.
Further, we choose the ratio, $\rho$, of where to choose the split point of subdivision.
For simplicity, we might want to subdivide at the half way point ($\rho = \frac{1}{2}$) but as we
will see, this might give lopsided sub-cuboid regions and using a better split point is desirable.

In what follows, our goal is to choose a subdivision that will reduce the average defect.
We assume that the start and end of the path lie in the $\alpha$ dimension.

Since the start and end path lie on the $\alpha$ dimension, we are forced to split in the $\alpha$ axis
to subdivide the cuboid.
In what follows, we always commit to splitting the $\alpha$ cuboid region by half, with a potentially
additional step to coerce the $A$ volume to a desired parity.

%%%%
%%%%
%%%%

\subsection{$|\alpha| >> |\beta| \sim |\gamma|$}

\begin{itemize}
  \item Shape template: $S_0$
  \item $\sigma = \frac{5}{3}$
  \item $\rho = \text{n/a}$
\end{itemize}


When $|\alpha|$ is much larger than $|\beta|$ or $|\gamma|$,
we split into two sub-cuboids along the $\alpha$ axis,
which we represent as an $S_0$ split.
As stated above, we pick the midway point for $\alpha$.
What's left is to decide how eccentric the cuboid should be to make the $S_0$ split.

Take $\sigma > 1$ and $s = \text{min}(|\beta|,|\gamma|)$, $\sigma s = |\alpha|$.
The defect of the original volume is:

$$
\begin{array}{ll}
  \lambda(|\alpha|,|\beta|,|\gamma|)  & = \frac{ |\alpha| \cdot |\beta| \cdot |\gamma| }{ \text{min}(|\alpha|,|\beta|,|\gamma|) } \\
                  & = \frac{ \sigma s \cdot s \cdot s }{ \text{min}(\sigma s,s,s) } \\
                  & = \sigma \\
\end{array}
$$

If $|\alpha| > 2 |\beta| = 2s$, then $\text{min}(\frac{|\alpha|}{2},|\beta|,|\gamma|) = s$ and we have an average defect $\lambda_s(V_1) = \frac{\sigma}{2}$.
Intuitively, we reduce the defect by splitting a long horizontal column into two parts, each of which is more cube like,
making it more cube-like on average.

We can get a better ratio of when to split by asking what the maximum value of $|\alpha| = \sigma s$ is when there is still an average defect reduction.
In this case, we restrict $|\alpha| < 2 |\beta|$, with $\text{min}(\frac{|\alpha|}{2},|\beta|,|\gamma|) = \frac{|\alpha|}{2}$,
the average defect becomes:

$$
\begin{array}{ll}
  \lambda_s(V_1) & = 2 (\frac{1}{2}) \frac{ \frac{\sigma}{2} \cdot s^3 }{ (\frac{\sigma}{2} s)^3 } \\
   & = \frac{4}{\sigma^2}
\end{array}
$$

We want to reduce the average defect relative to the original defect, so:

$$
\begin{array}{ll}
  & \lambda_s (V_1) < \lambda(V_0) \\
  \to & \frac{4}{\sigma^2} < \sigma \\
  \to & 4^{\frac{1}{3}} < \sigma \\ 
  \to & 4^{\frac{1}{3}} = 1.58740\dots < \sigma < \frac{5}{3} \\
\end{array}
$$

For any $\sigma > 4^{\frac{1}{3}}$, we know the average defect will be reduced.
For algorithmic simplicity, we've chosen the simple fraction $\frac{5}{3}$ (as $\frac{5}{3} > 4^{\frac{1}{3}}$)
for the eccentric $S_0$ split ratio.

%%%%
%%%%
%%%%

\subsection{$|\gamma| >> |\alpha| \sim |\beta|$ or $|\beta| << |\alpha| \sim |\gamma|$}


\begin{figure}[h]
  \centering
  \includegraphics[width=\linewidth]{gilbert3d_eccentric_appendix_s1.pdf}
  \caption{ An $S_1$ eccentric subdivision showing the threshold, $\sigma$, used to determine when we should use this split
            and the split point ratio, $\rho$, to reduce the defect for a more harmonious subdivision. }
  \label{fig:eccentric_s1}
\end{figure}



\begin{itemize}
  \item Shape template: $S_1$
  \item $\sigma = \frac{3}{2}$
  \item $\rho = \frac{1}{3}$
\end{itemize}

Here, we are trying to validate two parameters, $\sigma$ and $\rho$.
$\sigma$ is the ratio of when $|\gamma|$ (resp. $|\beta|$)
is larger (resp. smaller) to the minimum (resp. maximum) of the remaining side lengths as the threshold to trigger an $S_1$
template split.
$\rho$ is the proportion along the $\gamma$ (resp. $\beta$) direction of where to choose the $\gamma$ direction split.
The $S_1$ shape template has three subdivided cuboids, which we call $A$, $B$ and $C$, ordered by
the path traversal through the cuboids.

We will show that there is an average defect reduction when choosing $\sigma = \frac{3}{2}$ and
$\rho = \frac{1}{3}$.
If we take $s = \text{min}(|\alpha|, |\beta|)$ and $|\gamma| = \sigma s$, then the defect of the original volume is:

$$
\lambda(V_0) = \sigma
$$

From our choice of $\sigma$ and $\rho$, $\text{min}( \sigma \rho s, \frac{s}{2} ) = \frac{s}{2}$,
implying the minimum side length for $A$ and $C$ is $\frac{s}{2}$ and the minimum side length of $B$
is $\text{min}( \sigma (1-\rho) s, s) = s$.
From this, we can calculate the average defect of the subdivided cuboid:

$$
\begin{array}{ll}
  \lambda_s (V_1) & = 2 \cdot (\frac{1}{2}) \rho [ \frac{ \rho \sigma s \cdot \frac{s}{2} \cdot s }{ (\frac{s}{2})^3 } ] \\
        & + (1-\rho) [ \frac{ (1-\rho) \sigma s \cdot s \cdot s }{s^3} ] \\
   & = 4 \sigma \rho^2 + \sigma (1-\rho)^2 \\
  & = 4 \sigma \rho^2 + \sigma (1-\rho)^2 \\
  & = \frac{8}{9} \sigma = \frac{4}{3} < \frac{3}{2} \\
  \to & \lambda_s(V_1) < \lambda(V_0)
\end{array}
$$

Showing there is an average defect reduction.

%%%%
%%%%
%%%%


\subsection{$|\beta| >> |\alpha| \sim |\gamma|$ or $|\alpha| << |\beta| \sim |\gamma|$}

\begin{figure}[h]
  \centering
  \includegraphics[width=\linewidth]{gilbert3d_eccentric_appendix_s2.pdf}
  \caption{ An $S_2$ eccentric subdivision showing the threshold, $\sigma$, used to determine when we should use this split
            and the split point ratio, $\rho$, to reduce the defect for a more harmonious subdivision. }
  \label{fig:eccentric_s2}
\end{figure}

\begin{itemize}
  \item Shape template: $S_2$
  \item $\sigma = \frac{3}{2}$
  \item $\rho = \frac{1}{3}$
\end{itemize}

As with the previous case, we are trying to validate two parameters, $\sigma$, the amount
of eccentricity of one side length relative to the others, and $\rho$, where to subdivide the
cuboid along the $\beta$ axis.

Choosing $\sigma = \frac{3}{2}$ implies that the defect of the original volume is $\lambda(V_0) = \sigma = \frac{3}{2}$.
Further, $\text{min}( \sigma \rho s, \frac{s}{2} ) = \frac{s}{2}$, implying the $A$ and $C$ volumes
have minimum side length of $\frac{s}{2}$ and $\text{min}( (1-\rho) \sigma s, s) = s$, implying the
minimum side length of volume $C$ is $s$.

This allows us to calculate the average defect of the subdivided cuboid using an $S_2$ shape template:

$$
\begin{array}{ll}
  \lambda_s (V_1) & = 2 (\frac{1}{2}) \rho [ \frac{ \rho \sigma s \cdot \frac{s}{2} \cdot s }{(\frac{s}{2})^3} ] \\
  & + (1-\rho) \cdot [ \frac{ (1-\rho) \sigma s \cdot s \cdot s }{s^3} ] \\
  & = 4 \sigma \rho^2 + \sigma (1-\rho)^2 \\
  & = \frac{8}{9} \sigma = \frac{4}{3} < \frac{3}{2} \\
  \to & \lambda_s(V_1) < \lambda(V_0) \\
\end{array}
$$


