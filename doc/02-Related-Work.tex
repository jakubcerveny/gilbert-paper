\section{Related Work}

Peano was the first to discover the first space filling curve, with Hilbert later discovering
a space filling curve that we concern ourselves with in this paper \cite{hilbert2004david}
Hilbert's curve takes a one dimensional line and maps it to a compact, connected and locally connected
set.
For an introduction to space filling curves in addition to the necessary and sufficient
conditions for a space filling curve, the reader is referred to Sagan \cite{sagan_1994}.

As an artifact of the construction scheme, the Hilbert curve requires the side lengths to be exact powers of two.


%---
%
%Rong, Zhang and Lin provide a modified Hilbert curve to arbitrary rectangular cuboid regions,
%with an application towards image and video compression.
%  - subdivide by lopping off largest power of two
%  - can lead to long straight edges in normal conditions (7a) (?)
%  - non-unique
%  - complicated (?)
%  - only works on $W \times H \times 2^d$ (?!)
%
%Zhang, Kamata, Ueshige, "pseudo-Hilbert scan for arbitrary-sized arrays"
%  - 2d (only)
%  - complicated (?)
%  - doesn't generalize well to 3d (?)
%  - aeshetically not so nice (see fig 7c, 7d)  (?)
%
%Voorhies coherence measure (graphics gems) seems pretty simple and reasoable?
%
%Dafner, Cohen-Or and Matias use a \textit{context-based} space filling curve that
%uses an underlying image to ...????
%
%Lianyin etal. propose more efficient encoding and decoding algorithms for 3d Hilbert
%curves ...
%
%Haverkort enumerates unique 3d Hilbert curves ... but doesn't require subdivided
%cubes to be next to each other, so potentially has large skips/diagonal moves from
%one to another?
%
%Haverkort also creates an 'inventory' of 3d Hilbert curves that "share the same properties"
%as 2d Hilbert curves.
%  - looks to be restricted to powers of 2
%  - vertex continuous (which is what we want)
%  - has many definitions of locality and others that might be interesting
%
%Haverkort and Walderveen define some "locality and bounding box quality" metrics,
%presumably to measure sfcs
%
%Haverkort 'recursive tilings .. little frag' introduces Arrwwid numbers
%
%
%B{\"o}hm, Perdacher and Plant propose Hilbert traversal order for problems
%such as K-means, Cholesky decomposition and matrix multiplication so as to exploit
%better cache utilization from the the locality of the Hilbert curve.
%
%Gotsman and Lindenbaum quantify locality and show Hilbert curves are optimal in relation
%to their metric.
%
%Mokbel, Aref, Kamel define \textit{Jump, Contiguity, Reverse, Forward and Still}.
%Incorrectly call scan and sweep curves/paths space filling.
%
%
