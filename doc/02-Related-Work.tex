\section{Related Work}

%% this section is roughly organized as follows:
%%
%% * [x] first discovery (peano, hilbert)
%% * [x] defining feature (locality) and reference (sagan)
%% * [x] uses in "harder/erious" engineering (integration, data structures, etc.)
%% * [x] uses in visualization (soft uses)
%% * [x] generalizations
%%   - [x] rong etal and point out limitations (complicated, powers of two dim. etc.)
%%   - [x] tautenhahn, initial idea with bounds (2d only, but can see how to generalize to 3d)
%% * [x] enumeration of different sfc (tautenhahn has some of this but mostly should be haverkort)
%% * [x] measures of goodness
%%   - [x] haverkort mostly
%%   - [x] coherence (voorhies)
%% * close section with references to making it efficient
%%   - mention recursive soultion in this paper hasn't prioritized hyper efficient memory
%%     and rutime optimization
%%

Peano discovered the first space filling curve, with Hilbert later discovering
a space filling curve that we concern ourselves with in this paper \cite{hilbert2004david}.
Hilbert's curve takes a one dimensional line and maps it to a compact, connected and locally connected
set.
We only touch on the technical definitions in this paper but the reader is referred to
Sagan \cite{sagan_1994} for a more thorough introduction.
An introduction to the space filling curves with a focus on the 
Hahn-Mazurkiewicz theorem, the sufficient and necessary conditions to be a space
filling curve, can be found in Kupers \cite{kupers_2012}.

The locally connected property of space filling curves
provide a condition where
nearby index points in the domain lead to
scaled nearby points in the output embedded dimension.
The \textit{locality} feature means that nearby index points
cluster in the mapped higher dimensional space.
This feature can be used to highlight one dimensional clustered data in higher dimensions.

Wiener noticed that Lebesgue integration could be reduced from
higher dimensional integration to one-dimensional integration by the use of space filling curves \cite{wiener_1988}.
Lera and Sergeyev have investigated reducing higher dimensional global optimization
on functions satisfying the Lipschitz condition, reducing the multi dimensional optimization to a
single dimension through a space filling curve mapping \cite{lera_2010}.
Asano et al. discuss space filling curves and their uses in geometric data structures
to create localized spatial access patterns \cite{asano_1997},
B{\"o}hm, Perdacher and Plant
show speedups in Cholesky decomposition and the Floyd-Warshall algorithm by reducing cache misses from
a space filling curve data access schedule \cite{bohm_2021}.
Niedermeier discusses a new method of H-indexing as a benefit over a Hilbert indexing scheme to
create optimal locality for mesh indexing in \cite{niedermeier_2002}.

Munroe used the locality property to create a stylized mapping of the internet protocol (IPv4) space \cite{xkcd_195}.
Anders used a Hilbert curve mapping to visualize the human genome as a 2d map \cite{anders_2009}.
Cortesi has used a Hilbert curve to visualize binary files \cite{cortesi_2011}.
Dafner, Cohen-Or and Matias use a space filling curve that is adapted to the image context to improve autocorrelation
above what can be achieved with a Peano or Hilbert curve \cite{dafner_2000}.


Hamilton and Rau-Chaplin explore creating Hilbert curves with unequal side lengths in \cite{hamilton_2008} but
require the side lengths to still be exact powers of two.
Rong, Zhang and Lin \cite{rong_2021} proposed a generalization of the 2D Hilbert curve to arbitrary rectangular
regions that involves subdividing smaller regions into nearest powers of to areas until a base case is reached.
They extend their method to 3D but require the depth dimension to be an exact power of two.

In a website application and scanned note \cite{lutanho_2003}, Tautenhahn provided the basis for a generalized 2D
space filling curve.
Tautenhahn also included policies for when to subdivide regions preferentially in only one dimension, what we call
\textit{eccentric splits}.
Tautenhahn's exploration details the parity arguments necessary for when a subdivision scheme can be employed
without creating diagonal moves (\textit{notches}).
In this paper, we extend Tautenhahn's ideas to create a 2D and 3D generalized Hilbert curve, employing parity arguments
and conditions under which eccentric splits should be employed.

For recursive space filling curves, there is a choice in recursive or subdivision schemes.
Haverkort catalogs different techniques for constructing 3D Hilbert curves in \cite{haverkort_2016_inventory} and \cite{haverkort_2016_howmany}.


Though all space filling curves must fill the domain space and maintain locality, features or
fine grained qualities might differ between subdivision schemes.
Mokbel et al. provide some performance statistics of different recurrence schemes in \cite{mokbel_2002}.
Gotsman and Lindenbaum provide more detailed definitions of locality, explore their definition of locality for
different space filling curves and conclude that the Hilbert curve is close to optimal \cite{gotsman_1996}.
Haverkort and Walderveen provide a bounding-box quality metric for 2D space filling curves in \cite{haverkort_2010}.
In \cite{haverkort_2010_arrwwid}, Haverkort provides a concept of an Arrwwid number that measures how best a volume
can be covered by curve sections and discusses optimal choices of space filling curves based on this metric.
Voorhies defines the concept of \textit{coherence} to measure how many times a ray passes through a volume
to differentiate between zig-zag curves, Peano curves and Hilbert curves \cite{voorhies_1991}.

Though recursive subdivision methods can be made to run in $O(\log N)$ time and space ($N \sim W \times H \times D$),
this is undesirable under certain high performance conditions where a memory usage needs to be constant or
where constant bounds matter.
Butz described a non-recursive, byte-oriented algorithm for Hilbert's space filling curve in \cite{butz_1971}.
Moore later extended Butz's idea to create a fast non-recursive algorithm that includes range queries \cite{moore_2016}.
Jia et al. provide further enhancements for efficient the 3D Hilbert curve encoding and decoding in \cite{jia_2022}.
In Holzm{\"u}ller's bachelor thesis, an algorithm for average case $O(1)$ neighbor queries is shown \cite{holzmuller_2019}.

For a more general treatment on self avoiding walks, the reader is referred to Madras and Slate \cite{madras_2013}.
Though not directly connected to space filling curves,
the reader is referred to Christensen and Moloney \cite{christensen2005complexity} for methods used in data analysis
and visualization used in this paper.


%---
%
%Rong, Zhang and Lin provide a modified Hilbert curve to arbitrary rectangular cuboid regions,
%with an application towards image and video compression.
%  - subdivide by lopping off largest power of two
%  - can lead to long straight edges in normal conditions (7a) (?)
%  - non-unique
%  - complicated (?)
%  - only works on $W \times H \times 2^d$ (?!)
%
%Zhang, Kamata, Ueshige, "pseudo-Hilbert scan for arbitrary-sized arrays"
%  - 2d (only)
%  - complicated (?)
%  - doesn't generalize well to 3d (?)
%  - aeshetically not so nice (see fig 7c, 7d)  (?)
%
%Voorhies coherence measure (graphics gems) seems pretty simple and reasoable?
%
%Dafner, Cohen-Or and Matias use a \textit{context-based} space filling curve that
%uses an underlying image to ...????
%
%Lianyin etal. propose more efficient encoding and decoding algorithms for 3d Hilbert
%curves ...
%
%Haverkort enumerates unique 3d Hilbert curves ... but doesn't require subdivided
%cubes to be next to each other, so potentially has large skips/diagonal moves from
%one to another?
%
%Haverkort also creates an 'inventory' of 3d Hilbert curves that "share the same properties"
%as 2d Hilbert curves.
%  - looks to be restricted to powers of 2
%  - vertex continuous (which is what we want)
%  - has many definitions of locality and others that might be interesting
%
%Haverkort and Walderveen define some "locality and bounding box quality" metrics,
%presumably to measure sfcs
%
%Haverkort 'recursive tilings .. little frag' introduces Arrwwid numbers
%
%
%B{\"o}hm, Perdacher and Plant propose Hilbert traversal order for problems
%such as K-means, Cholesky decomposition and matrix multiplication so as to exploit
%better cache utilization from the the locality of the Hilbert curve.
%
%Gotsman and Lindenbaum quantify locality and show Hilbert curves are optimal in relation
%to their metric.
%
%Mokbel, Aref, Kamel define \textit{Jump, Contiguity, Reverse, Forward and Still}.
%Incorrectly call scan and sweep curves/paths space filling.
%
%
