\section{Algorithm}

\subsection{Overview}

We discuss one possible generalization of the Hilbert curve, which we call the Gilbert curve,
to arbitrary axis-aligned cuboid regions.

The design of the Gilbert curve is done to provide a:

\begin{itemize}
  \item Conceptually simple algorithm
  \item Harmonious realization
  \item Resulting curve with good locality conditions
\end{itemize}

The discretized Hilbert curve recursively traces out a Hamiltonian path through a square
region whose sides are powers of two.
The Gilbert curve extends this idea to trace out Hamiltonian paths through to arbitrary axis-aligned
cuboid regions.

The Gilbert curve recursively subdivides a larger cuboid region into smaller cuboid
regions of different sizes and orientations.
The subdivision scheme uses a shape template that will be explained in more detail in subsequent sections.
The orientations of the subdivided cuboid regions are chosen so that the recursive path
will start and end at pre-specified endpoints, connecting neighboring cuboid regions
after a path is realized.

If a cuboid becomes too oblong, or \textit{eccentric}, a subdivision shape template scheme is chosen
in an attempt to create subdivisions that are closer to being cube like.

%The cuboid region is represented by an origion point, $p \in \mathbb{Z}^3$,
%and a local coordinate system of three vectors $\alpha, \beta, \gamma \in \mathbb{Z}^3$.

The subdivided cuboid region is processed by the Gilbert curve algorithm,
with virtual origin point $p \in \mathbb{Z}^3$ and local coordinate system $[ \alpha, \beta, \gamma ]$,
with $\alpha, \beta, \gamma \in \mathbb{Z}^3$.
Each of $\alpha$, $\beta$ and $\gamma$ are axis aligned and orthogonal.

In a cuboid region, the the Hamiltonian path start at $p$ and ends at $p + \alpha$.
For this reason, we call $\alpha$ the ``width-like'' dimension, with $\beta$ and $\gamma$
called the ``height-like'' dimension and ``depth-like'' dimension, respectively.

When processing the subdivided cuboid regions, the coordinate system is updated
with new integral lengths and rotated.
Since rotations are only in units of $(\pi/2)$ radians and $\alpha$, $\beta$, $\gamma$ start
off as axis-aligned and orthogonal, they remain axis aligned and orthogonal throughout.

%Using parity arguments, the possibility of a Hamiltonian path it's not difficult to show that
%if a Hamiltonian path is possible or not.
%for a given side lengths along with a start and end point,
%combined with certain cuboid side lengths,
When applying the shape template to choose the subdivided cuboid regions during the coarse of the
algorithm, side lengths are chosen so as to preserve the possibility of a Hamiltonian path.
Should the lengths of the cuboid being subdivided preclude a Hamiltonian path from occurring, the lengths
of the subdivided cuboids are chosen
so that all but one will preserve the possibility of a Hamiltonian path.
In this case, since only a single cuboid absorbs the impossibility of a Hamiltonian path,
and this process is recursively applied, there will be a single jump, or \textit{notch},
in the resulting curve, globally.

%Each of $\alpha$, $\beta$ and $\gamma$ start off as axis-aligned vectors of integral
%length and are only rotated by units of $(\pi/2)$ radians, remaining axis aligned throughout.
%$\alpha$, $\beta$ and $\gamma$ represent the dimensions of the sub-cuboid region
%and the local basis when tracing out a curve.

The next sub-section discusses parity arguments for when a valid curve can be
traced out in a sub-region.
We then discuss in detailed the 2D Gilbert curve algorithm and end with a discussion of the 3D Gilbert curve.

When talking about the 2D Gilbert curves, we assume vectors are in $\mathbb{Z}^3$
so they can be used without alteration for algorithms working in 3D.

%In what follows, a path is assumed to start at the local reference point $(0,0,0)$
%and end at the far edge of the the width like dimension, $\alpha$ (e.g. $(w-1,0,0)$).

\subsection{Valid Paths from Grid Parity}

\begin{figure}[h]
  \centering
  \includegraphics[width=\linewidth]{simple_hampath.pdf}
  \caption{ Illustrative examples of Hamiltonian paths height/width that are even/even, even/odd and odd/odd, respectively,
            when starting from the lower left hand corner }
  \label{fig:exampleHampath}
\end{figure}


The feasibility of determining whether there exists a Hamiltonian path in a rectangular cuboid
grid region can be accomplished through parity arguments.
Label grid cell points in a volume as 0 or 1,
alternating between labels with every axis-aligned single step move.
Any Hamiltonian path that ends at one of the three remaining corners has to have the same parity as the starting point if the
volume is odd, or different parity if the volume is even.

\begin{table}[h]
  \centering
  \begin{tabular}[t]{cr|cc}
    \multicolumn{2}{c}{ \multirow{2}{*}{Path Possible} } & \multicolumn{2}{c}{Volume} \\
    & & \textit{even} & \textit{odd} \\
    \hline
			%\multirow{2}{*}{$|\alpha| \bmod 2$} & \textit{even} & Yes & Yes \\
      \multirow{2}{*}{ $|\alpha|$ } & \textit{even} & Yes & Yes \\
       & \textit{odd} & \textbf{No} & Yes \\
     \hline
  \end{tabular}
  \caption{ Table showing when a Hamiltonian path is possible. Here, $|\alpha|$ is the absolute difference in start and end position of the path which
            coincides with one of the axis aligned side length of the cuboid volume. A Hamiltonian path is not possible on only in the case
            when $|\alpha|$ is odd and the volume is even. }
  \label{table:pathTable}
\end{table}


For a path starting at $(0,0,0)$ and ending $|\alpha|$ steps in one of the axis-aligned dimensions,
then Table \ref{table:pathTable} enumerates this condition under which a valid path is possible.
Figure \ref{fig:exampleHampath} illustrates this for starting position $(0,0)$ with volumes $(2 \times 2)$, $(3 \times 2)$ and $(3 \times 3)$,
where a red cross indicating a precluded endpoint.

Without loss of generality, we will assume a curve starts from position $p_s=(0,0,0)$ and has proposed
endpoint at $p_e=((w-1),0,0)$, with a cuboid region as $\alpha = (w,0,0), \beta = (0,h,0), \gamma = (0,0,d)$.
We state, without proof, that
a Hamiltonian path is always possible from $p_s$ to $p_e$ when $|\alpha|$ is even or when $|\alpha|$, $|\beta|$ and $|\gamma|$
are all odd  $(|\alpha| \cdot (1 - |\beta| \cdot |\gamma|) \equiv 0 \bmod 2)$.

With this condition, any cuboid subdivision will always have a Hamiltonian path within it and we can recreate a Hamiltonian
path in the larger cuboid by connecting endpoints from the ending point of one cuboid to the starting point of the succeeding cuboid.
For cuboids that violate this condition, there will be a required notch and
an admissible cuboid subdivision is possible
for all but one of the cuboid subdivisions, recursively limiting the notch violation to a single point.


%%%%%%%%%%%%%%%%%%%%%%%%%%%%%%%%%%%%%%%%%%%%%%%%
%            _ ____              __ ___       __
%     ____ _(_) / /_  ___  _____/ /|__ \ ____/ /
%    / __ `/ / / __ \/ _ \/ ___/ __/_/ // __  / 
%   / /_/ / / / /_/ /  __/ /  / /_/ __// /_/ /  
%   \__, /_/_/_.___/\___/_/   \__/____/\__,_/   
%  /____/                                       
%%%%%%%%%%%%%%%%%%%%%%%%%%%%%%%%%%%%%%%%%%%%%%%%


\subsection{2D Generalized Hilbert Function (Gilbert2D)}

\begin{figure}[h]
  \centering
  \includegraphics[width=\linewidth]{gilbert2d_mainsubdiv.pdf}
  \caption{ For the 2D Gilbert curve, the main rectangle is subdivided into three sub regions, making sure their paths connect
            and preferring even side lengths for the first rectangular subdivision.
            Partitioning the rectangle in this way is what we call a $U$-split.  }
  \label{fig:main2dsubdiv}
\end{figure}


\begin{figure*}[ht]
  \centering
  \includegraphics[width=\linewidth]{config_production.pdf}
  \caption{ Enumeration of the subdivision template depending on different parities of $\alpha$ and $\beta$ dimensions. }
  \label{fig:production2d}
\end{figure*}

For the 2D Gilbert curve, a $U$-split template is used, highlighted in Figure \ref{fig:main2dsubdiv}.
The $U$-split breaks the region into three sub-blocks, labeled $A$, $B$ and $C$.
Each of the $A$, $B$ and $C$ regions have local coordinate systems that are resized, rotated and
new endpoints chosen so the process can be recursively re-applied.

The sides of the subdivided blocks $A$, $B$ and $C$ are chosen to remain integral.
The width-like dimension for subdivided block $A$ is chosen to be even ($\beta_2$ in Figure \ref{fig:main2dsubdiv})
by dividing the height-like dimension of the original region ($\beta$ in Figure \ref{fig:main2dsubdiv}) by two and by adding one if need be.
Since the width-like length of the subdivided $A$ and $C$ block is even ($\beta_2$), there is a guaranteed valid Hamiltonian path 
in both.

Should the original width-like length be even ($\alpha$ in Figure \ref{fig:main2dsubdiv}), the $C$ block will continue to have 
a valid Hamiltonian path.
If the original width-like dimension is odd, a Hamiltonian path is not possible and there is a forced notch that will appear in the subdivided block $C$.

Figure \ref{fig:production2d} gives examples of the different parity conditions for the original area,
as well as the different conditions when the $A$ and $C$ subdivided areas are coerced into having even parity.
Figure \ref{fig:production2d} also shows when a Hamiltonian
path is not possible, indicated by a red cross, and illustrates how the notch is pushed into the $C$ block
under these conditions.

If a subdivided block becomes too long in its width-like dimension, the block is divided in half and the recursion
proceeds as normal.
Even sides are only enforced if the length is larger than 2, with a length 2 or smaller representing the base
case.
The base case enumerates paths in the axis direction of length greater than one.

% redundant...
Since $A$ and $C$ are chosen to have an even width-like local axis length, no new notches are introduced and any
obligatory notch is effectively ``pushed'' into the $C$ block.


%The $A$ block is chosen to to have a preference for even height dimension


\begin{algorithm}
  \caption{\hskip0.5em 2D Generalized Hilbert Function (Gilbert2D) } %($p$, $\alpha$, $\beta$) \\ \hskip3.0em $p, \alpha, \beta \in \mathbb{Z}^3$ }
  \label{alg:gilbert2d}
  \begin{algorithmic}

    \State \textit{\# $p, \alpha, \beta \in \mathbb{Z}^3$}
    \Function{Gilbert2D}{$p$, $\alpha$, $\beta$}
      \State
      \State $\alpha_2, \beta_2  = (\alpha // 2), (\beta // 2)$

      \State
      \If{ $(|\beta| \equiv 1)$ }
        \State \textbf{yield} $p + i \cdot \delta(\alpha)$ \textbf{forall} $i \in |\alpha|$

        \State
      \ElsIf{ $(|\alpha| \equiv 1)$ }
        \State \textbf{yield} $p + i \cdot \delta(\alpha)$ \textbf{forall} $i \in |\beta|$

        \State
      \ElsIf{ $(2 |\alpha| > 3 |\beta|)$ }
        \If{ $(|\alpha_2| > 2)$ and $(|\alpha_2| \bmod{2} \equiv 1)$ }
          \State $\alpha_2 \leftarrow \alpha_2 + \delta(\alpha)$
        \EndIf
        \State
        \State \textbf{yield} Gilbert2D($p$, \\ \hskip9.75em $\alpha_2$, $\beta$)
        \State
        \State \textbf{yield} Gilbert2D($p + \alpha_2$, \\ \hskip9.75em $\alpha - \alpha_2$, $\beta$)

        \State
      \Else
        \If{ $(|\beta_2| > 2)$ and $(|\beta_2| \bmod{2} \equiv 1)$ }
          \State $\beta_2 \leftarrow \beta_2 + \delta(\beta)$
        \EndIf
        \State
        \State \textbf{yield} Gilbert2D($p$, \\ \hskip9.75em $\beta_2$, $\alpha_2$)
        \State
        \State \textbf{yield} Gilbert2D($p + \beta_2$, \\ \hskip9.75em $\alpha$, $(\beta - \beta_2)$)
        \State
        \State \textbf{yield} Gilbert2D($p + \alpha - \delta(\alpha) + \beta_2 - \delta(\beta)$, \\ \hskip9.75em $\beta_2$, $-(\alpha - \alpha_2)$)
      \EndIf
    \EndFunction
  \end{algorithmic}
\end{algorithm}

Algorithm \ref{alg:gilbert2d} shows the pseudo-code for computing the 2D Gilbert curve.
Note that $\alpha$ and $\beta$ are taken to be vectors in 3D, where the third dimension
can be ignored if a purely 2D curve is desired.
The generalization to 3D allows the Gilbert2D function to be used unaltered when the
3D Gilbert curve needs to trace out in-plane sub-curves.

%\floatname{algorithm}{Procedure}
%\begin{algorithm}
%  \caption{\hskip0.5em $\delta(\cdot)$ directional vector function }
%  \label{alg:delta}
%  \begin{algorithmic}
%
%    \State
%    \State \textit{\# integral sign function}
%    \Function{$\text{sgn}$}{$w \in \mathbb{Z}$}
%      %\State $(w < 0) ? (-1) : ((w > 0) ? 1 : 0)$
%      \State \textbf{if} $(w < 0)$ \Return $-1$
%      \State \textbf{if} $(w > 0)$ \Return $1$
%      \State \Return 0
%    \EndFunction
%
%    \State
%    \State \textit{\# directional vector}
%    \Function{$\delta$}{$v \in \mathbb{Z}^3$}
%      \State \Return $[ \text{sgn}(v_0), \text{sgn}(v_1), \text{sgn}(v_2) ]$
%    \EndFunction
%
%  \end{algorithmic}
%\end{algorithm}
%\floatname{algorithm}{Algorithm}

The $\delta(\cdot)$ function returns one of the six directional vectors indicating which of
the major signed axis aligned directions the input vector points to ($[\pm1,0,0], [0,\pm1,0],[0,0,\pm1]$).
For completeness, the function is defined in Procedure \ref{alg:delta}.

Algorithm \ref{alg:gilbert2d} assumes standard Euclidean two norm ($|v| = \sqrt{v_0^2 + v_1^2 + v_2^2}$)
and abuses notation by allowing scalar to vector multiplication ($i \in \mathbb{Z}, v \in \mathbb{Z}^3, i \cdot v \to [ i \cdot v_0, i \cdot v_1, i \cdot v_2 ]$).
where the context is clear.

Example outputs of the \textit{Gilbert2D} function are listed in Figure \ref{fig:gilbert2d_examples}, showing a $(4 \times 4)$, $(10 \times 4)$ and $(7 \times 4)$

%%%%%%%%%%%%%%%%%%%%%%%%%%%%%%%%%%%%%%%%%%%%%%%%%
%            _ ____              __ _____     __
%     ____ _(_) / /_  ___  _____/ /|__  /____/ /
%    / __ `/ / / __ \/ _ \/ ___/ __//_ </ __  / 
%   / /_/ / / / /_/ /  __/ /  / /____/ / /_/ /  
%   \__, /_/_/_.___/\___/_/   \__/____/\__,_/   
%  /____/                             
%%%%%%%%%%%%%%%%%%%%%%%%%%%%%%%%%%%%%%%%%%%%%%%%%

\subsection{3D Generalized Hilbert Function (Gilbert3D)}


\begin{figure}[h]
  \centering
  \includegraphics[width=\linewidth]{gilbert3d_explode.pdf}
  \caption{ The $J_0$-split subdivision, representing the main subdivision of the bulk recursion for the 3D Gilbert curve case. }
  \label{fig:gilbert3DJSplit}
\end{figure}

\begin{figure}[h]
  \centering
  \includegraphics[width=\linewidth]{gilbert3d_eccentric.pdf}
  \caption{ Eccentric cases for recursion. When the cuboid is too far from being cube-like, an $S$-split shape template is used to try and subdivide the cuboids into elements that are more cube like. Here $\alpha$, $\beta$, $\gamma$ are the width-like, height-like and depth-like dimensions. See the text or the algorithm description for the constant ratios used to determine what makes the  cuboid lengths eccentric. }
  \label{fig:gilbert3DEccentricCase}
\end{figure}


The concept of a $U$-split is extended into 3D by constructing a $J$-split.
An exploded view of a $J$-split is shown in Figure \ref{fig:gilbert3DJSplit}.

Side lengths of the subdivided cuboids are chosen to try and not preclude a
Hamiltonian path.
If a path starts at the local $[0,0,0]$, a Hamiltonian path that ends
at $[(w-1),0,0]$ is only possible when $w$ is even or when all lengths are odd, including $w$.


\begin{figure*}[ht]
  \centering
  \includegraphics[width=\textwidth]{gilbert3d_case.pdf}
  \caption{ Bulk recursion J-split atlas for the 3D Gilbert algorithm }
  \label{fig:gilbert3DCase}
\end{figure*}

%\floatname{algorithm}{Procedure}
%\begin{algorithm}
%  \caption{ \hskip0.5em $S_0$-Split function (eccentric split) }
%  \label{alg:procS0}
%  \begin{algorithmic}
%    \State
%    \State \textit{\# split halfway on $\alpha$ }
%    \Function{$S_0$}{$p$, $\alpha$, $\beta$, $\gamma$}
%      \State $\alpha_2 \leftarrow (\alpha // 2)$
%      \If{ $(|\alpha| > 2)$ and $((|\alpha_2| \bmod{2}) \equiv 1)$ }
%        \State $\alpha_2 \leftarrow \alpha_2 + \delta(\alpha)$
%      \EndIf
%      \State
%      \State \textbf{yield} Gilbert3D($p$, \\ \hskip8.25em $\alpha_{2}$, $\beta$, $\gamma$ )
%      \State
%      \State \textbf{yield} Gilbert3D($p + \alpha_{2}$, \\ \hskip8.25em $(\alpha - \alpha_{2})$, $\beta$, $\gamma$ )
%    \EndFunction
%  \end{algorithmic}
%\end{algorithm}
%\floatname{algorithm}{Algorithm}
%
%
%\floatname{algorithm}{Procedure}
%\begin{algorithm}
%  \caption{ \hskip0.5em $S_1$-Split functions (eccentric split) }
%  \label{alg:procS1}
%  \begin{algorithmic}
%    \State
%    \State \textit{\# split $\frac{1}{3}$ on $\gamma$ and halfway on $\alpha$ }
%    \Function{$S_1$}{$p$, $\alpha$, $\beta$, $\gamma$}
%      \State $\alpha_2, \gamma_3 \leftarrow (\alpha // 2), (\gamma // 3)$
%      \If{ $(|\alpha| > 2)$ and $((|\alpha_2| \bmod{2}) \equiv 1)$ }
%        \State $\alpha_2 \leftarrow \alpha_2 + \delta(\alpha)$
%      \EndIf
%      \If{ $(|\gamma| > 2)$ and $((|\gamma_3| \bmod{2}) \equiv 1)$ }
%        \State $\gamma_3 \leftarrow \gamma_3 + \delta(\gamma)$
%      \EndIf
%      \State
%      \State \textbf{yield} Gilbert3D($p$, \\ \hskip8.25em $\gamma_3$, $\alpha_2$, $\beta$)
%      \State
%      \State \textbf{yield} Gilbert3D($p + \gamma_3$, \\ \hskip8.25em $\alpha$, $\beta$, $(\gamma - \gamma_3)$)
%      \State
%      \State \textbf{yield} Gilbert3D($p + \alpha - \delta(\alpha) + \gamma_{3} - \delta(\gamma)$, \\ \hskip8.25em $\gamma_3$, $(\alpha - \alpha_2)$, $\beta$)
%    \EndFunction
%  \end{algorithmic}
%\end{algorithm}
%\floatname{algorithm}{Algorithm}
%
%
%\floatname{algorithm}{Procedure}
%\begin{algorithm}
%  \caption{ \hskip0.5em $S_2$-Split function (eccentric split) }
%  \label{alg:procS2}
%  \begin{algorithmic}
%    \State
%    \State \textit{\# split $\frac{1}{3}$ on $\beta$ and halfway on $\alpha$ }
%    \Function{$S_2$}{$p$, $\alpha$, $\beta$, $\gamma$}
%      \State $\alpha_2, \beta_3 \leftarrow (\alpha // 2), (\beta // 3)$
%      \If{ $(|\alpha| > 2)$ and $((|\alpha_2| \bmod{2}) \equiv 1)$ }
%        \State $\alpha_2 \leftarrow \alpha_2 + \delta(\alpha)$
%      \EndIf
%      \If{ $(|\beta| > 2)$ and $((|\beta_3| \bmod{2}) \equiv 1)$ }
%        \State $\beta_3 \leftarrow \beta_3 + \delta(\beta)$
%      \EndIf
%      \State
%      \State \textbf{yield} Gilbert3D($p$, \\ \hskip8.25em $\beta_{3}$, $\gamma$, $\alpha_2$ )
%      \State
%      \State \textbf{yield} Gilbert3D($p + \beta_{3}$, \\ \hskip8.25em $\alpha$, $(\beta - \beta_{3})$, $\gamma$ )
%      \State
%      \State \textbf{yield} Gilbert3D($p + \alpha - \delta(\alpha) + \beta_{3} - \delta(\beta)$, \\ \hskip8.25em $-\beta_{3}$, $\gamma$, $-\alpha$)
%    \EndFunction
%  \end{algorithmic}
%\end{algorithm}
%\floatname{algorithm}{Algorithm}
%
%
%\floatname{algorithm}{Procedure}
%\begin{algorithm}
%  \caption{ \hskip0.5em $J_0$-Split function }
%  \label{alg:procJ0}
%  \begin{algorithmic}
%    \State
%    \State \textit{\# $|\gamma|$ even }
%    \Function{$J_0$}{$p$, $\alpha$, $\beta$, $\gamma$}
%      \State
%      \State $\alpha_2, \beta_2, \gamma_2 \leftarrow (\alpha // 2), (\beta // 2), (\gamma // 2)$
%      \State
%      \State \textit{\# prefer initial block even}
%      \State $\alpha_2=\alpha_2+\delta(\alpha)$ \textbf{if} $(|\alpha_2| > 2)$and$(|\alpha_2| \bmod{2} \equiv 1)$
%      \State $\beta_2=\beta_2+\delta(\beta)$ \textbf{if} $(|\beta_2| > 2)$and$(|\beta_2| \bmod{2} \equiv 1)$
%      \State $\gamma_2=\gamma_2+\delta(\gamma)$ \textbf{if} $(|\gamma_2| > 2)$and$(|\gamma_2| \bmod{2} \equiv 1)$
%      \State
%      \State \textbf{yield} Gilbert3D($p$, \\ \hskip8.25em $\beta_2$, $\gamma_2$, $\alpha_2$)
%      \State
%      \State \textbf{yield} Gilbert3D($p+\beta_2$, \\ \hskip8.25em $\gamma$, $\alpha_2$, $\beta - \beta_2$)
%      \State
%      \State \textbf{yield} Gilbert3D($p + \beta_2 - \delta(\beta_2) + \gamma - \delta(\gamma)$, \\ \hskip8.25em $\alpha$, $-\beta_2$, $-(\gamma - \gamma_2)$)
%      \State
%      \State \textbf{yield} Gilbert3D($p + \alpha - \delta(\alpha) + \beta_2 + \gamma - \delta(\gamma)$, \\ \hskip8.25em $-\gamma$, $-(\alpha - \alpha_2)$, $(\beta - \beta_2)$)
%      \State
%      \State \textbf{yield} Gilbert3D($p + \alpha - \delta(\alpha) + \beta_2 - \delta(\beta)$, \\ \hskip8.25em $-\beta_2$, $\gamma_2$, $-(\alpha - \alpha_2)$)
%    \EndFunction
%  \end{algorithmic}
%\end{algorithm}
%\floatname{algorithm}{Algorithm}


%\floatname{algorithm}{Procedure}
%\begin{algorithm}
%  \caption{ \hskip0.5em $J_1$-Split function }
%  \label{alg:procJ2}
%  \begin{algorithmic}
%    \State
%    \State \textit{\# $|\gamma|$ odd, one of $|\alpha|$ or $|\beta|$ even }
%    \Function{$J_1$}{$p$, $\alpha$, $\beta$, $\gamma$}
%      \State
%      \State $\alpha_2, \beta_2, \gamma_2 \leftarrow (\alpha // 2), (\beta // 2), (\gamma // 2)$
%      \State
%      \State \textit{\# prefer $\beta_2$, $\gamma_2$ even but force $\alpha_2$ odd }
%      \State $\alpha_2=\alpha_2+\delta(\alpha)$ \textbf{if} $(|\alpha_2| > 2)$and$(|\alpha_2| \bmod{2} \equiv 0)$
%      \State $\beta_2=\beta_2+\delta(\beta)$ \textbf{if} $(|\beta_2| > 2)$and$(|\beta_2| \bmod{2} \equiv 1)$
%      \State $\gamma_2=\gamma_2+\delta(\gamma)$ \textbf{if} $(|\gamma_2| > 2)$and$(|\gamma_2| \bmod{2} \equiv 1)$
%      \State
%      \State \textbf{yield} Gilbert3D($p$, \\ \hskip8.25em $\gamma2$, $\alpha_2$, $\beta_2$)
%      \State
%      \State \textbf{yield} Gilbert3D($p+\gamma_2$, \\ \hskip8.25em $\beta$, $\gamma - \gamma_2$, $\alpha_2$)
%      \State
%      \State \textbf{yield} Gilbert3D($p+\gamma_2 - \delta(\gamma) + \beta - \delta(\beta)$, \\ \hskip8.25em $\alpha$, $-(\beta - \beta_2)$, $-\gamma_2$)
%      \State
%      \State \textbf{yield} Gilbert3D($p+\alpha - \delta(\alpha) + \beta - \delta(\beta) + \gamma_2 - \delta(\gamma)$, \\ \hskip8.25em $\beta$, $\gamma - \gamma_2$, $-(\alpha - \alpha_2)$)
%      \State
%      \State \textbf{yield} Gilbert3D($p+\alpha - \delta(\alpha) + \gamma_2 - \delta(\gamma)$, \\ \hskip8.25em $-\gamma_2$, $-(\alpha - \alpha_2)$, $\beta_2$)
%    \EndFunction
%  \end{algorithmic}
%\end{algorithm}
%\floatname{algorithm}{Algorithm}


%\floatname{algorithm}{Procedure}
%\begin{algorithm}
%  \caption{ \hskip0.5em $J_2$-Split function }
%  \label{alg:procJ2}
%  \begin{algorithmic}
%    \State
%    \State \textit{\# $|\alpha|, |\beta|, |\gamma|$ odd }
%    \Function{$J_2$}{$p$, $\alpha$, $\beta$, $\gamma$}
%      \State
%      \State $\alpha_2, \beta_2, \gamma_2 \leftarrow (\alpha // 2), (\beta // 2), (\gamma // 2)$
%      \State
%      \State \textit{\# prefer $\beta_2$, $\gamma_2$ even but force $\alpha_2$ odd }
%      \State $\alpha_2=\alpha_2+\delta(\alpha)$ \textbf{if} $(|\alpha_2| > 2)$and$(|\alpha_2| \bmod{2} \equiv 0)$
%      \State $\beta_2=\beta_2+\delta(\beta)$ \textbf{if} $(|\beta_2| > 2)$and$(|\beta_2| \bmod{2} \equiv 1)$
%      \State $\gamma_2=\gamma_2+\delta(\gamma)$ \textbf{if} $(|\gamma_2| > 2)$and$(|\gamma_2| \bmod{2} \equiv 1)$
%      \State
%      \State \textbf{yield} Gilbert3D($p$, \\ \hskip8.25em $\beta_2$, $\gamma$, $\alpha_2$)
%      \State
%      \State \textbf{yield} Gilbert3D($p+\beta_2$, \\ \hskip8.25em $\gamma_2$, $\alpha$, $(\beta - \beta_2)$)
%      \State
%      \State \textbf{yield} Gilbert3D($p+\beta_2 + \gamma_2$, \\ \hskip8.25em $\alpha$, $(\beta - \beta_2)$, $(\gamma - \gamma_2)$)
%      \State
%      \State \textbf{yield} Gilbert3D($p + \alpha - \delta(\alpha) + \beta_2 - \delta(\beta) + \gamma_2$, \\ \hskip8.25em $-\beta_2$, $(\gamma- \gamma_2)$, $-(\alpha - \delta(\alpha))$)
%      \State
%      \State \textbf{yield} Gilbert3D($p + \alpha - \delta(\alpha) + \gamma_2 - \delta(\gamma)$, \\ \hskip8.25em $-\gamma_2$, $-(\alpha - \alpha_2)$, $\beta_2$)
%    \EndFunction
%  \end{algorithmic}
%\end{algorithm}
%\floatname{algorithm}{Algorithm}


\begin{algorithm}
  \caption{ \hskip0.5em 3D Generalized Hilbert Function (Gilbert3D) } %($p$, $\alpha$, $\beta$, $\gamma$) \\ \hskip3.0em $p, \alpha, \beta, \gamma \in \mathbb{Z}^3$ }
  \label{alg:gilbert3d}
  \begin{algorithmic}
    \State \textit{\# $p, \alpha, \beta, \gamma \in \mathbb{Z}^3$}
    \Function{Gilbert3D}{$p$, $\alpha$, $\beta$, $\gamma$}

    \State
    \State\textit{\# Parity of dimensions}

    \State $\alpha_0 \leftarrow (|\alpha|\bmod{2})$
    \State $\beta_0 \leftarrow (|\beta|\bmod{2})$
    \State $\gamma_0 \leftarrow (|\gamma|\bmod{2})$

    \State
    \State\textit{\# Base cases }
    \State \textbf{if} ($(|\alpha|\equiv 2)$ and $(|\beta|\equiv 2)$ and $(|\gamma| \equiv 2)$)
    \State \hskip1.5em \Return Hilbert3D($p$,$\alpha$,$\beta$,$\gamma$) 
    \State \Return Gilbert2D($p$,$\beta$,$\gamma$) \textbf{if} $(|\alpha| \equiv 1)$
    \State \Return Gilbert2D($p$,$\alpha$,$\gamma$) \textbf{if} $(|\beta| \equiv 1)$
    \State \Return Gilbert2D($p$,$\alpha$,$\beta$) \textbf{if} $(|\gamma| \equiv 1)$

    \State
    \State\textit{\# Eccentric cases }

    \State \textbf{if }$(3 |\alpha|>5|\beta|) \text{ and } (3|\alpha|>5|\gamma|))$
    \State \hskip1.5em \Return $S _ 0$($p$,$\alpha$,$\beta$,$\gamma$) 
    \State \textbf{if }$(2 |\beta| > 3 |\gamma|) \text{ or }(2 |\beta| > 3 |\alpha|))$
    \State \hskip1.5em \Return $S _ 2$($p$,$\alpha$,$\beta$,$\gamma$)
    \State \textbf{if }$(2 |\gamma| > 3 |\beta|)$
    \State \hskip1.5em \Return $S _ 1$($p$,$\alpha$,$\beta$,$\gamma$)

    \State
    \State \textit{\# Bulk recursion }
    \State \Return $J _ 0$($p$,$\alpha$,$\beta$,$\gamma$) \textbf{if} $(\gamma_0 \equiv 0)$
    \State \Return $J _ 1$($p$,$\alpha$,$\beta$,$\gamma$) \textbf{if} $(\alpha_0 \equiv 0)$or$(\beta_0\equiv 0)$
    \State \Return $J _ 2$($p$,$\alpha$,$\beta$,$\gamma$)

    \EndFunction

  \end{algorithmic}
\end{algorithm}


